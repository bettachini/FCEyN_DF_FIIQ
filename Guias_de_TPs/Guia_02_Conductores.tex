\documentclass[problemas]{guia}

\def \practnum {2} 
\def \practica {Conductores, Capacidad, Condensadores, Diel\'ectricos, %
                Polarizaci\'on, Campos \vec{E} y \vec{D} }

\def \materia {Laboratorio de Fisica II para Quimicos}
\def \periodo {2do. Cuatrimestre de 2015}
\def \catedra {P. Cobelli}
 
% \usepackage{graphics}
% \usepackage{amsmath}
\usepackage{amsfonts}
\usepackage{graphicx}
\usepackage[arrowdel]{physics}
\usepackage{float}
\usepackage{wrapfig}
\usepackage{subfigure}
% \usepackage{bm}
% \usepackage{grffile}
\usepackage{color}
% \usepackage{framed}
\usepackage[utf8]{inputenc}
% \usepackage[T1]{fontenc}
% \usepackage{lmodern}

\usepackage[separate-uncertainty=true, multi-part-units=single, locale=FR]{siunitx}

\usepackage{tikz}
\usepackage[siunitx]{circuitikz}

% circuitikz: rotate instruments - https://tex.stackexchange.com/questions/105864/rotate-voltmeter-circuitikz
\newcommand{\mymeter}[2] 
{  % #1 = name , #2 = rotation angle
\begin{scope}[transform shape,rotate=#2]
\draw[thick] (#1)node(){$\mathbf V$} circle (11pt);
\draw[rotate=45,-latex] (#1)  +(-17pt,0) --+(17pt,0);
\end{scope}
}

\usepackage{isotope} % $\isotope[A][Z]{X}\to\isotope[A-4][Z-2]{Y}+\isotope[4][2]{\alpha}$

\usepackage[spanish]{babel}
\usepackage{babelbib}
\selectbiblanguage{spanish}
 
\usepackage{enumerate}
% definicion del entorno 'sabermas'
\makeatletter
\definecolor{shadecolor}{rgb}{0.89,0.91,0.94}
\newenvironment{sabermas}[1]{%
\vfill
\begin{shaded}
  \begin{center}
  {\textsection{Para saber m\'as}}
  \end{center}
  #1
\sf } 
{%
\end{shaded}%
}
\makeatother

\renewcommand{\vec}[1]{\ensuremath{\mathbf{#1}}}



\input{hyphenation_rules}
% \graphicspath{{./guia8/}}

\begin{document} 
\maketitle

\begin{problema}{}
    Dentro de un conductor hueco de forma arbitraria, se encuentra alojado un 
    segundo conductor. Se carga a uno de ellos con carga $Q$ y al otro con 
    carga $Q'$. ¿Sobre cuáles superficies se distribuyen las cargas?
    ¿Qué ocurre si ambos conductores se tocan?
    Muestre que si $Q' = -Q$, entonces el campo exterior es nulo.
\end{problema}

\begin{problema}{}
    Un conductor esférico, hueco y sin cargas tiene un radio interior $a$ y 
    otro exterior $b$. En el centro de la esfera se encuentra una carga puntual
    $+q$. ¿Cómo es la distribución de cargas? Calcule y grafique el campo  
    eléctrico y el potencial en todos los puntos del espacio.
\end{problema}

\begin{problema}{}
    Calcular la capacidad de las siguientes configuraciones de conductores:
    \begin{enumerate}[(a)]
        \item una esfera de radio $R$ en el vacío; determinar el valor de $R$ 
            que haga $C = 1$ pF.
        \item un condensador esférico de radio interior $a$ y exterior $b$. 
            Comparar con el resultado anterior para $b$ muy grande.
        \item por unidad de longitud, para un condensador cilíndrico infinito 
            de radios $R_1$ y $R_2$.
        \item por unidad de área, para un condensador plano infinito; si la 
            separación entre placas es de 1 mm, dar el valor del área para 
            que $C = 1 F$.
    \end{enumerate}
\end{problema}

\begin{problema}{}
    Una esfera conductora de radio $a$ está rodeada por un casquete esférico, 
    también conductor, de radio interior $b$ y exterior $c$. Ambos conductores
    se encuentran unidos por un cable y su carga total es $Q$. En el espacio 
    entre ambos se encuentra una superficie esférica de radio $d$ 
    ($a < d < b$), cargada con una densidad superficial de carga $\sigma$. 
    Calcule el campo eléctrico en todo el espacio (considere que el cable no 
    rompe la simetría esférica del problema).
\end{problema}

\begin{problema}{} 
    Un condensador de 1 $\mu$F soporta tensiones no mayores de 6 kV, y otro 
    de 2 $\mu$F, no superiores a 4 kV. ¿Qué tensión soportan si se los 
    conecta en serie?
\end{problema}

\begin{problema}{}
    Cuatro capacitores idénticos están conectados a una batería $V_0$ como se 
    muestra en la figura. Al comenzar, la llave 1 está cerrada y la llave 2 
    está abierta. Luego de un tiempo muy largo se abre la llave 1 y se cierra 
    la llave 2. ¿Cuál será la diferencia de potencial final entre los 
    capacitores si la batería es de 9V?
\end{problema}

\begin{problema}{}
    En el circuito de la figura:
    \begin{enumerate}[(a)]
        \item Calcule la capacidad equivalente que se observa desde la batería.
        \item Encuentre  las  cargas  de  cada  condensador  y  calcule  la 
            energía del sistema.
        \item Se desconecta la batería. ¿Se redistribuyen las cargas?
        \item Si ahora agregamos un dieléctrico lineal de permitividad 
            $\varepsilon$ en el condensador $C_1$, ¿cómo se redistribuyen las 
            cargas? ¿Cuál es la energía del sistema? ¿Dónde está la energía 
            que falta?
    \end{enumerate}
\end{problema}

\begin{problema}{}
    Entre las placas de un capacitor plano de sección $A$ se coloca un 
    dieléctrico como muestra la figura de arriba. Posteriormente se carga 
    hasta que adquiere una carga $Q$ y se lo desconecta de la fuente.
    \begin{enumerate}[(a)]
        \item Determine el valor de la capacidad del sistema, la diferencia de
            potencial entre las placas y la energía acumulada en el capacitor.
        \item ¿Qué sucederá con la carga, la diferencia de potencial y la 
            energía si se le retira el dieléctrico? ¿Y si no se hubiese 
            desconectado la fuente?
        \item Repita los cálculos anteriores para el caso en que el dieléctrico
            se coloca como muestra la figura de abajo.
    \end{enumerate}
\end{problema}

\begin{problema}{}
    Entre las placas de un capacitor plano se colocan dos materiales 
    dieléctricos de constantes $\varepsilon_1$ y $\varepsilon_2$ como se 
    muestra en la figura. Halle la capacidad, considerando que no existen 
    cargas libres en la interface entre los dieléctricos.
\end{problema}

\begin{problema}{}
    Una esfera cargada uniformemente con carga $Q$ fue instalada en el seno de
    un dieléctrico de constante dieléctrica $\varepsilon$. Determine la carga 
    de polarización en la interface entre el dieléctrico y la esfera.
\end{problema} 

\end{document}
