\documentclass[problemas]{guia}

\def \practnum {5} 
\def \practica {Corrientes variables, ley de Faraday y ley de Lenz}

\def \materia {Laboratorio de Fisica II para Quimicos}
\def \periodo {2do. Cuatrimestre de 2015}
\def \catedra {P. Cobelli}
 
% \usepackage{graphics}
% \usepackage{amsmath}
\usepackage{amsfonts}
\usepackage{graphicx}
\usepackage[arrowdel]{physics}
\usepackage{float}
\usepackage{wrapfig}
\usepackage{subfigure}
% \usepackage{bm}
% \usepackage{grffile}
\usepackage{color}
% \usepackage{framed}
\usepackage[utf8]{inputenc}
% \usepackage[T1]{fontenc}
% \usepackage{lmodern}

\usepackage[separate-uncertainty=true, multi-part-units=single, locale=FR]{siunitx}

\usepackage{tikz}
\usepackage[siunitx]{circuitikz}

% circuitikz: rotate instruments - https://tex.stackexchange.com/questions/105864/rotate-voltmeter-circuitikz
\newcommand{\mymeter}[2] 
{  % #1 = name , #2 = rotation angle
\begin{scope}[transform shape,rotate=#2]
\draw[thick] (#1)node(){$\mathbf V$} circle (11pt);
\draw[rotate=45,-latex] (#1)  +(-17pt,0) --+(17pt,0);
\end{scope}
}

\usepackage{isotope} % $\isotope[A][Z]{X}\to\isotope[A-4][Z-2]{Y}+\isotope[4][2]{\alpha}$

\usepackage[spanish]{babel}
\usepackage{babelbib}
\selectbiblanguage{spanish}
 
\usepackage{enumerate}
% definicion del entorno 'sabermas'
\makeatletter
\definecolor{shadecolor}{rgb}{0.89,0.91,0.94}
\newenvironment{sabermas}[1]{%
\vfill
\begin{shaded}
  \begin{center}
  {\textsection{Para saber m\'as}}
  \end{center}
  #1
\sf } 
{%
\end{shaded}%
}
\makeatother

\renewcommand{\vec}[1]{\ensuremath{\mathbf{#1}}}



\input{hyphenation_rules}
% \graphicspath{{./guia8/}}

\begin{document} 
\maketitle

\begin{problema}{}
    Una espira circular de 1000 vueltas y 100 cm$^2$ de área está colocada en 
    un campo magnético uniforme de $0.01$ T y rota 10 veces por segundo en 
    torno de uno de sus diámetros que es normal a la dirección del campo. 
    Calcular:
    \begin{enumerate}[(a)]
        \item la fuerza electromotriz (f.e.m.) inducida en la espira en función 
            del tiempo $t$ y, en particular, cuando su normal forma un ángulo 
            de 45$^\circ$ con el campo.
        \item la f.e.m. máxima y mínima y los valores de $t$ para que aparezcan
            estas f.e.m.
    \end{enumerate}
\end{problema} 

\begin{problema}{}
    En la figura se muestra un disco de Faraday, consistente en un disco de 
    cobre de radio $a$ cuyo eje es paralelo a un campo magnético uniforme 
    $\vec{B}$. Si  el  disco  rota  con  una  velocidad  angular  $\omega$,  
    calcular  la  f.e.m.  que aparece entre los puntos A y C.
\end{problema} 

\begin{problema}{}
    Los rieles de una vía están separados por $1.5$ m y están aislados entre 
    sí. Se conecta entre ellos un milivoltímetro. ¿Cuánto indica el instrumento
    cuando pasa un tren a 200 km/h? Suponer que la componente vertical del 
    campo magnético de la Tierra mide allí $1.5\times10^{-5}$ T.
\end{problema} 

\begin{problema}{}
    Un cable rectilíneo muy largo conduce una corriente $I$ de 1 A. A una 
    distancia $L = 1$ m del cable se encuentra el extremo de una aguja de 
    40 cm de largo que gira en torno de ese extremo en el plano del cable, con 
    una velocidad angular $\omega = XXXX$ rad/s, como se muestra en la figura. 
    Calcular la f.e.m. inducida en los extremos de la aguja como función del 
    tiempo.
\end{problema} 

\begin{problema}{}
    Un solenoide tiene 1000 vueltas, 20 cm de diámetro y 40 cm de largo. En su 
    centro se ubica otro solenoide de 100 vueltas, 4 cm de diámetro y espesor 
    despreciable, cuya resistencia vale 50 $\Omega$. Si la corriente que 
    circula por el solenoide exterior aumenta a razón de $0.5$ A cada $0.2$ s,
    calcular la corriente que se induce en el solenoide interior, cuya 
    autoinductancia es de $2.4$ mH.
\end{problema} 

\begin{problema}{}
    Calcular la autoinductancia de:
    \begin{enumerate}[(a)]
        \item un solenoide infinito de radio $R$ y $n$ vueltas por unidad de 
            longitud (exprese el resultado por unidad de longitud).
        \item un toroide con $N$ vueltas, sección $S$ y radio medio $R$, usando
            que la diferencia entre el radio exterior e interior es mucho menor
            que $R$.
        \item un solenoide de longitud $L$ y radio $R$ (suponga $R \ll L$), 
            con $N$ vueltas.
    \end{enumerate}
\end{problema} 

\begin{problema}{}
    Calcule la energía magnética por unidad de longitud para el cable coaxil 
    del Problema 10 de la Guía 4. Utilizando la relación entre la energía y la 
    autoinductancia, encuentre esta última.
\end{problema} 

\begin{problema}{}
    Dos cables rectilíneos paralelos de radio $r$, separados por una distancia 
    $d$, pueden suponerse como un circuito que se cierra por el infinito. 
    Encuentre la autoinductancia por unidad de longitud cuando $r \ll d$.
\end{problema} 

\begin{problema}{}
    Calcule $M_{12}$ y $M_{21}$ entre una espira circular de radio $R$ y un 
    solenoide finito de longitud $L$ y radio $r$ (suponga que 
    $r \ll L$ y $r \ll R$), dispuestos de tal forma que los centros y los ejes 
    de ambos son coincidentes. Utilice las aproximaciones que crea necesarias 
    y diga cuál de los dos resultados es más confiable cuando $L$ es pequeño 
    respecto a $R$.
\end{problema} 

\begin{problema}{}
    Dos bobinas están conectadas en serie a una distancia tal que la mitad del 
    flujo de una de ellas atraviesa también la otra. Si la autoinducción de 
    las bobinas es $L$, calcular la autoinducción del conjunto, suponiendo que 
    las bobinas están conectadas de tal forma que los flujos se suman.
\end{problema} 

\begin{problema}{}
    Un condensador de 3 $\mu$F se carga a $271.8$ V y luego se descarga a 
    través de una resistencia de 1 M$\Omega$. Calcular:
    \begin{enumerate}[(a)]
        \item el voltaje sobre el condensador luego de 3 segundos.
        \item el calor disipado en la resistencia durante la descarga completa 
            del condensador. Comparar el valor obtenido con la energía 
            almacenada en el condensador al comienzo de la descarga.
    \end{enumerate}
\end{problema} 

\begin{problema}{}
    La figura muestra las condiciones del circuito antes de $t=0$, instante en
    que se cierra la llave S. Calcular para todo instante $t > 0$:
    \begin{enumerate}[(a)]
        \item El voltaje sobre el condensador $C_2$.
        \item La corriente.
    \end{enumerate}
\end{problema} 

\begin{problema}{}
    Una f.e.m. de 400 V se conecta en tiempo $t = 0$ a un circuito serie 
    formado por una inductancia $L = 2$ H, una resistencia $R = 20$ $\Omega$
    y un capacitor $C = 8$ $\mu$F inicialmente descargado.
    \begin{enumerate}[(a)]
        \item Demostrar que el proceso de carga es oscilatorio y calcular la 
            frecuencia de las oscilaciones. Comparar esta frecuencia con el 
            valor de $(LC)^{-1/2}$.
        \item Calcular la derivada temporal inicial de la corriente.
        \item Hallar, en forma aproximada, la máxima tensión sobre $C$.
        \item ¿Qué resistencia debe agregarse en serie para que el 
            amortiguamiento del circuito sea crítico?
    \end{enumerate}
\end{problema} 

\begin{problema}{}
    Considere el circuito que se muestra en la figura. Todas las resistencias 
    son iguales. En el instante $t_0=0$ se cierra la llave que conecta al 
    capacitor. Calcule en cuánto tiempo a partir de $t_0$ el capacitor habrá 
    alcanzado el 99\% de su carga máxima, suponiendo que inicialmente estaba 
    descargado, e indique cuál será su polaridad. Ayuda: reduzca el circuito a
    sólo dos mallas.
\end{problema} 

\end{document}
