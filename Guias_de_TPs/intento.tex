

Guía 1: Electrostática



Problema 10:
Una corona circular de radios a y b tiene una densidad de carga uniforme σ.
a)	Hallar el campo eléctrico en su eje.
b)	Deducir del resultado anterior el campo eléctrico en el eje de un disco de radio b y luego el campo eléctrico de un plano, ambos cargados uniformemente. En cada caso estudie la continuidad del campo y obtenga el valor del “salto” en la discontinuidad.

Problema 11:
En cada uno de los casos siguientes determine, explotando la simetría de la configuración de cargas, cuál será la dirección del campo eléctrico y de cuáles coordenadas dependerán sus componentes. Utilizando el teorema de Gauss determine el campo eléctrico en todo el espacio, y a partir de éste calcule el potencial electrostático. Grafique las líneas de campo y las superficies equipotenciales.
a)	Un hilo delgado infinito con densidad lineal uniforme λ.
b)	Un cilindro circular infinito de radio R, cargado uniformemente en volumen con densidad
ρ.
c)	Un plano infinito con densidad superficial de carga uniforme σ.
d)	Una esfera de radio R con densidad uniforme ρ.
e)	Una esfera de radio R con densidad de carga ρ = A rn (A, n = constantes)

Nota: Observe que en los tres primeros casos no se puede tomar el cero de potencial en el infinito ni se lo puede calcular mediante la integral:
V (r) = k	ρ (r′) d3r′ + constante
r − r′

ya que ella no está definida para esas distribuciones de carga.
 


Problema 12:
Calcule la integral definida en el problema anterior para la situación descripta en el Problema 9. Verifique que su gradiente es –E. ¿Qué ocurre cuando la longitud del hilo se hace infinita? Nota: Dado que estamos calculando el potencial sólo para puntos sobre un plano perpendicular al hilo y que pasa por el centro del mismo, el resultado no sirve para obtener la componente del campo  eléctrico  perpendicular  a  ese  plano.  Sin  embargo,  por  simetría  sabemos  que  esa componente debe ser nula.

Problema 13:
En ciertas condiciones, el campo eléctrico de la atmósfera apunta hacia la superficie de la Tierra. Sobre la superficie su valor es de 300 V/m, mientras que a 1400 m de altura, es de 20 V/m.
a)	Calcule la carga total contenida en un volumen cilíndrico vertical cuya base está sobre la superficie terrestre y su altura es de 1400 m. ¿Cuál es la carga media por unidad de volumen en esa región de la atmósfera? (Suponga que el problema es plano).
b)	En la atmósfera podemos encontrar iones negativos y positivos. Suponiendo que el valor absoluto de la carga de cada ion es e = 1.6×10-19 C, escriba la densidad de carga como función de n- y n+ (número de iones negativos y positivos por unidad de volumen).

Problema 14:
Una molécula de agua tiene su átomo de oxígeno en el origen y los núcleos de hidrógeno en x = ( ±0.077 nm; 0.058 nm). Si los electrones del hidrógeno se transfieren completamente al átomo de oxígeno, ¿cuál sería el momento dipolar de la molécula? Compare con el valor experimental (esta caracterización de los enlaces químicos del agua como totalmente iónicos sobrestima el momento dipolar).

Problema 15:
Un anillo de radio R está cargado uniformemente con una carga total –q. En el centro del mismo se coloca una carga puntual q.
a)	¿Cuánto valen los momentos monopolar y dipolar? ¿Depende el momento dipolar del origen de coordenadas?
b)	Calcule el potencial y el campo eléctrico sobre el eje del anillo y estudie el comportamiento a distancias grandes.
 



Guía 2: Conductores, Capacidad, Condensadores, Dieléctricos, Polarización, Campos E y D.

Problema 1:
Dentro de un conductor hueco de forma arbitraria, se encuentra alojado un segundo conductor. Se carga a uno de ellos con carga Q y al otro con carga Q'.
¿Sobre cuáles superficies se distribuyen las cargas?
¿Qué ocurre si ambos conductores se tocan?
Muestre que si Q' = −Q, entonces el campo exterior es nulo.



Problema 2:
Un conductor esférico, hueco y sin cargas tiene un radio interior a y otro exterior b. En el centro de la esfera se encuentra una carga puntual +q. ¿Cómo es la distribución de cargas? Calcule y grafique el campo  eléctrico y el potencial en todos los puntos del espacio.



 
Problema 3:
En un campo eléctrico uniforme cargado con carga total Q.
 


E0    se introduce un cuerpo conductor de forma arbitraria
 
a)	¿Qué valor tiene la fuerza eléctrica que se ejerce sobre el cuerpo?
b)	Como consecuencia de la inducción de cargas sobre la superficie del conductor, el campo dejará de ser uniforme en la vecindad del cuerpo. Si se "congela'' la distribución superficial de carga y se quita el campo externo, ¿cómo será el campo en el interior del cuerpo? Notar que al congelar la carga superficial, el cuerpo pierde las propiedades de un conductor.



Problema 4:
Calcular la capacidad de las siguientes configuraciones de conductores:
a)	una esfera de radio R en el vacío; determinar el valor de R que haga C = 1 pF.
b)	un condensador esférico de radio interior a y exterior b. Comparar con el resultado anterior para b muy grande.
c)	por unidad de longitud, para un condensador cilíndrico infinito de radios R1 y R2.
d)	por unidad de área, para un condensador plano infinito; si la separación entre placas es de 1
mm, dar el valor del área para que C = 1 F.



Problema 5:
Una esfera conductora de radio a está rodeada por un casquete esférico, también conductor, de radio interior b y exterior c. Ambos conductores se encuentran unidos por un cable y su carga total es Q. En el espacio entre ambos se encuentra una superficie esférica de radio d (a < d < b), cargada con una densidad superficial de carga σ. Calcule el campo eléctrico en todo el espacio (considere que el cable no rompe la simetría esférica del problema).



Problema 6:
Un condensador de 1 µF soporta tensiones no mayores de 6 kV, y otro de 2 µF, no superiores a 4 kV. ¿Qué tensión soportan si se los conecta en serie?
 




Problema 7:
Cuatro capacitores idénticos están conectados a una batería V0 como se muestra en la figura. Al comenzar, la llave 1 está cerrada y la llave 2 está abierta. Luego de un tiempo muy largo se abre la llave 1 y se cierra la llave 2. ¿Cuál será la diferencia de potencial final entre los capacitores si la batería es de 9V?




Problema 8:
En el circuito de la figura:
a)	Calcule la capacidad equivalente que se observa desde la batería.
b)	Encuentre  las  cargas  de  cada  condensador  y  calcule  la energía del sistema.
c)	Se desconecta la batería. ¿Se redistribuyen las cargas?
d)	Si ahora agregamos un dieléctrico lineal de permitividad ε en el condensador C1, ¿cómo se redistribuyen las cargas? ¿Cuál es la energía del sistema? ¿Dónde está la energía que falta?



Problema 9:
Entre las placas de un capacitor plano de sección A se coloca un dieléctrico como muestra la figura de arriba. Posteriormente se carga hasta que adquiere una carga Q y se lo desconecta de la fuente.
a)	Determine el valor de la capacidad del sistema, la diferencia de potencial entre las placas y la energía acumulada en el capacitor.
b)	¿Qué sucederá con la carga, la diferencia de potencial y la energía si se le retira el dieléctrico? ¿Y si no se hubiese desconectado la fuente?
c)	Repita los cálculos anteriores para el caso en que el dieléctrico se coloca como muestra la figura de abajo.




Problema 10:
Entre las placas de un capacitor plano se colocan dos materiales dieléctricos de constantes ε1 y ε2  como se muestra en la figura. Halle la capacidad, considerando que no existen cargas libres en la interface entre los dieléctricos.



Problema 11:
Una esfera cargada uniformemente con carga Q fue instalada en el seno de un dieléctrico de constante dieléctrica ε. Determine la carga de polarización en la interface entre el dieléctrico y la esfera.
 


Guía 3: Corrientes estacionarias, ley de Ohm, teorema de Thevenin, transferencia de potencia, conexiones de resistencias.

Problema 1:
Un cable de cobre (resistividad del Cu: 1.7×10-8 Ω·m) de 2 mm de radio y 1 m de longitud se estira hasta cuadruplicar su longitud (las secciones inicial y final son uniformes).
a)	Calcular la resistencia antes y después del estiramiento, suponiendo que la resistividad no varía.
b)	Por el cable de cobre de 2 mm2 de sección circula una corriente de 1 A. Si hay un electrón
de conducción por cada átomo, encuentre la velocidad media de los electrones. Datos: δCu = 9 g/cm3, e = 1.60×10-19 C, Na = 6×1023/mol, ACu = 63.5.
c)	Calcular  la  resistencia  eléctrica  de  una  plancha,  una  estufa  de  cuarzo,  una  lamparita
eléctrica de 60 W y una lamparita de linterna.



Problema 2:
Para el circuito representado en la figura de la derecha:
a)	Calcular las corrientes de ramas y de mallas.
b)	Repetir después de cambiar una de las resistencias de 12 Ω por una de 6 Ω.
c)	Calcular la potencia disipada por cada resistencia y la entregada
por  la  fuente  en  los  puntos  anteriores.  Verificar  que  la  condición  para  la  máxima transferencia de potencia se cumple.
d)	Calcular el consumo en kWh luego de dos días de funcionamiento en los dos casos.



 

 

interna del amperímetro es Ra = 1 Ω?
 
Problema 3:
Para el circuito que muestra la figura de la izquierda, calcular:
a)	las corrientes i1 e i2.
b)	la diferencia de potencial entre los puntos C y D.
c)	la potencia disipada por las resistencias de 5 Ω.
d)	Se coloca un amperímetro en serie con la batería
de 20 V.	¿Qué corriente mide si la resistencia
 
e)	Repita el punto anterior pero ahora considerando que el amperímetro está en serie con la
resistencia de 3 Ω.
f)	Comparar los dos puntos anteriores con el primero.



Problema 4:
En el circuito de la figura calcular:
a)	la resistencia equivalente vista desde la fuente.
b)	la corriente i y la caída de potencial entre los puntos B y C.
c)	la potencia entregada por la fuente.
 


Problema 5:
Determinar  la  potencia  suministrada  a  una  resistencia  que  se conecta entre A y B si su valor es:
a)  R1=1 Ω. b)  R2=5 Ω. c)   R3=10 Ω.
d)  R4 tal que la transferencia de potencia resulte máxima.

Problema 6:
a)	Obtener el circuito equivalente de Thevenin para el puente de la figura (conocido como puente de Wheatstone) visto desde los puntos A y B.
b)	Entre A y B se conecta un galvanómetro de resistencia interna R. Calcular la corriente que circula por él en función de ε, R1, R 2, R 3, R 4 y R.
c)	Determine  la  relación  entre  las  resistencias  para  la cual la
corriente que circula por el amperímetro es nula. Ésta se llama condición de equilibrio del puente y se emplea para medir resistencias con precisión.
d)	Hallar  la  potencia  disipada  por  el  galvanómetro  cuando:
ε = 1 V, R4 = 1.1 Ω, R1 = R2 = R3 = 1Ω y R = 0.1 Ω
 


Guía 4: Magnetostática, Ley de Ampère, vectores B y H

Problema 1:
Una partícula de carga q se mueve en un campo magnético uniforme B con una velocidad v
perpendicular al campo.
a)	Calcule el radio de la órbita circular descripta. ¿Aumenta el módulo de la velocidad? ¿Por qué?
b)	Determine la frecuencia del movimiento circular descripto.
c)	¿Qué sucede si la velocidad es paralela al campo magnético? ¿Y si tiene una componente paralela al campo y otra perpendicular?

Problema 2:
Una partícula de carga q entra, con una velocidad v , en una región del espacio donde existe un campo eléctrico uniforme E de 80 kV/m dirigido hacia abajo, como se muestra en la figura. Perpendicular al campo eléctrico y a la velocidad de la partícula cargada, se halla un campo
magnético B de 0,4T. Si la rapidez de la partícula se escoge apropiadamente, ésta no sufrirá ninguna deflexión a causa de los campos perpendiculares. ¿Qué rapidez debe ser seleccionada en este caso? (Este dispositivo se llama selector de velocidades)

Problema 3:
La figura muestra un dispositivo empleado para la medición de la masa de los iones. Un ion de masa m y carga +q sale esencialmente en reposo de la fuente S, cámara donde se produce la descarga de un gas. La diferencia de potencial V acelera el ion y se permite que entre  en  una  región  con  un  campo  magnético  perpendicular
uniforme B . Dentro del campo, el ion se mueve en semicírculo, chocando con una placa fotográfica a la distancia x de la rendija de entrada. Demuestre que la masa m del ion está dada por:
2
 
m = B
 
q x2
 

Problema 4:
 
8 ⋅V
 
Calcule la fuerza por unidad de longitud entre dos cables paralelos por los que circula una corriente de 30A. La separación entre cables es de 2 cm. Estime hasta qué distancia por encima de los cables se verá afectada la indicación de una brújula. Considere los dos posibles sentidos de circulación de la corriente. (Suponga que la intensidad del campo magnético terrestre en el lugar es de 5×10-4 T y forma un ángulo de 30º con la vertical).

Problema 5:
a)	Calcule el campo magnético sobre el eje de una espira circular de área A y corriente I.
b)	Repita el cálculo para una espira cuadrada.
c)	Estudie  y compare  los  comportamientos  de  ambos  resultados  para  distancias  grandes.
Expréselos en función de los momentos magnéticos de las espiras.

Problema 6:
a)	Calcule el campo magnético sobre el eje de un solenoide de longitud L, con N vueltas devanadas densamente, por el que circula una corriente I.
b)	Estudie  el  comportamiento  a  grandes  distancias  y  encuentre  el  valor  del  momento magnético del solenoide.
 


c)	Obtenga el límite de solenoide infinito.
d)	Suponga que el solenoide tiene 40 cm de largo, 10 cm de diámetro y el campo en el centro es de 3 T (éste es un campo muy intenso). Si el solenoide se encuentra en el subsuelo del pabellón I, ¿influirá en la medición del campo magnético terrestre que realizan los alumnos en el segundo piso?

Problema 7:
Calcule la fuerza sobre una aguja pequeña magnetizada con momento magnético m, colocada sobre el eje del solenoide finito del problema anterior. Exprese la fuerza en función de la distancia al centro del solenoide. Discuta el sentido de la fuerza en relación a los sentidos del momento magnético m y el campo magnético B .

Problema 8:
Dibuje cualitativamente las líneas de campo magnético correspondientes a dos cables rectilíneos infinitos y paralelos, que conducen sendas corrientes I de sentido contrario. Tenga en cuenta cuál debe ser el comportamiento del campo cerca y lejos de los cables.

Problema 9:
Aprovechando la simetría de la distribución de corrientes y usando la ley de Ampère, determine el vector campo magnético en los siguientes casos:
a)	un cable rectilíneo infinito por el que circula una corriente I.
b)	un cilindro infinito de radio R por el que circula una densidad de corriente uniforme  j .
c)	un solenoide infinito de n vueltas por unidad de longitud y corriente I (suponga que el devanado es suficientemente denso como para despreciar la componente longitudinal de los elementos de corriente).
d)	un plano infinito con densidad superficial de corriente g uniforme.
e)	dos  planos  infinitos  paralelos,  separados  una  distancia  d,  con  densidades  de  corriente uniformes g y - g .
f)	una lámina infinita de caras plano-paralelas y espesor d, con densidad de corriente   j
uniforme.
g)	un toroide de radio interior a y radio exterior b, con un arrollamiento denso de N vueltas por el que circula una corriente I.

Problema 10:
Un cable coaxil está formado por dos conductores cilíndricos coaxiales separados por un medio de permeabilidad µ (ver figura). Por ambos conductores circulan corrientes I iguales y opuestas. Suponiendo que la densidad  de corriente en  cada uno  de los  conductores  es  uniforme,
encuentre el campo magnético B en todo punto del espacio.



Problema 11:
Un cilindro infinito de radio a es circulado por una corriente volumétrica uniforme  j  = j0   z , coaxial con el cilindro. En la zona b < r < c (a < b), se tiene un medio magnético lineal,
isótropo y homogéneo cuya permeabilidad relativa es µr =1000.

 
a)	Calcular los campos H
 
y B en todo el espacio.
 
b)	¿Se comporta el medio material como un blindaje magnético?
 


Guía 5: Corrientes Variables, ley de Faraday, ley de Lenz, coeficientes de inducción, energía magnética, períodos transitorios.

Problema 1:
Una espira circular de 1000 vueltas y 100 cm2 de área está colocada en un campo magnético uniforme de 0.01 T y rota 10 veces por segundo en torno de uno de sus diámetros que es normal a la dirección del campo. Calcular:
a)	la f.e.m. inducida en la espira en función del tiempo t y, en particular, cuando su normal forma un ángulo de 45° con el campo.
b)	la f.e.m. máxima y mínima y los valores de t para que aparezcan estas f.e.m.



Problema 2:
En la figura se muestra un disco de Faraday, consistente en un disco de cobre de radio a cuyo eje es paralelo a un campo magnético uniforme B .
Si  el  disco  rota  con  una  velocidad  angular  ω,  calcular  la  f.e.m.  que
aparece entre los puntos A y C.



Problema 3:
Los rieles de una vía están separados por 1.5 m y están aislados entre sí. Se conecta entre ellos un milivoltímetro. ¿Cuánto indica el instrumento cuando pasa un tren a 200 km/h? (Considere que esto pasa en Francia o en Alemania donde tal fenómeno es posible). Suponer que la componente vertical del campo magnético de la Tierra mide allí  1.5×10-5 T.



Problema 4:
Un cable rectilíneo muy largo conduce una corriente I de 1A. A una distancia L = 1 m del cable se encuentra el extremo de una aguja de 40 cm de largo que gira en torno de ese extremo en el plano del cable, con una velocidad angular ω = 20π s-1, como se muestra en la figura. Calcular la f.e.m. inducida en los extremos de la aguja como función del tiempo.



Problema 5:
Un solenoide tiene 1000 vueltas, 20 cm de diámetro y 40 cm de largo. En su centro se ubica otro solenoide de 100 vueltas, 4 cm de diámetro y espesor despreciable, cuya resistencia vale 50 Ω. Si la corriente que circula por el solenoide exterior aumenta a razón de 0,5 A cada 0.2 s, calcular la corriente que se induce en el solenoide interior, cuya autoinductancia es de 2,4 mH.

Problema 6:
Calcular la autoinductancia de:
a)	un solenoide infinito de radio R y n vueltas por unidad de longitud (exprese el resultado por unidad de longitud).
b)	un toroide con N vueltas, sección S y radio medio R, usando que la diferencia entre el radio exterior e interior es mucho menor que R.
c)	un solenoide de longitud L y radio R (suponga R << L), con N vueltas.
 



Problema 7:
Calcule la energía magnética por unidad de longitud para el cable coaxil del Problema 10 de la Guía 4. Utilizando la relación entre la energía y la autoinductancia, encuentre esta última.

Problema 8:
Dos cables rectilíneos paralelos de radio r, separados por una distancia d, pueden suponerse como un circuito que se cierra por el infinito. Encuentre la autoinductancia por unidad de longitud cuando r << d.



Problema 9:
Calcule M12 y M21 entre una espira circular de radio R y un solenoide finito de longitud L y radio r (suponga r << L y r << R), dispuestos de tal forma que los centros y los ejes de ambos son coincidentes. Utilice las aproximaciones que crea necesarias y diga cuál de los dos resultados es más confiable cuando L es pequeño respecto a R.

Problema 10:
Dos bobinas están conectadas en serie a una distancia tal que la mitad del flujo de una de ellas atraviesa también la otra. Si la autoinducción de las bobinas es L, calcular la autoinducción del conjunto, suponiendo que las bobinas están conectadas de tal forma que los flujos se suman.



Problema 11:
Un condensador de 3 µF se carga a 271.8 V y luego se descarga a través de una resistencia de 1MΩ. Calcular:
a)	el voltaje sobre el condensador luego de 3 segundos.
b)	el calor disipado en la resistencia durante la descarga completa del condensador. Comparar el valor obtenido con la energía almacenada en el condensador al comienzo de la descarga.



Problema 12:
La figura muestra las condiciones del circuito antes de t=0, instante en que se cierra la llave S. Calcular para toda t > 0:
a)	El voltaje sobre el condensador C2.
b)	La corriente.



Problema 13:
Una f.e.m. de 400 V se conecta en tiempo t = 0 a un circuito serie formado por una inductancia
L = 2 H, una resistencia R = 20 Ω y un capacitor C = 8 µF inicialmente descargado.
a)	Demostrar que el proceso de carga es oscilatorio y calcular la frecuencia de las oscilaciones.
Comparar esta frecuencia con el valor de (LC)-1/2.
b)	Calcular la derivada temporal inicial de la corriente.
c)	Hallar, en forma aproximada, la máxima tensión sobre C.
d)	¿Qué resistencia debe agregarse en serie para que el amortiguamiento del circuito sea crítico?
 


 
Problema 14:
Considere el circuito que se muestra en la figura. Todas las resistencias son iguales. En el instante t0=0 se cierra la llave que conecta al capacitor. Calcule en cuánto tiempo a partir de t0 el capacitor habrá alcanzado el 99% de su carga máxima, suponiendo que inicialmente estaba descargado, e indique cuál será su polaridad.
Ayuda: reduzca el circuito a sólo dos mallas.
 



R2

V	R1

C
R3	R4
R6	R5
 


Guía 6: Circuitos de Corriente Alterna

Problema 1:
Un condensador C = 1µF está conectado en paralelo con una inductancia L = 0.1 H cuya resistencia interna vale R = 1Ω. Se conecta la combinación a una fuente alterna de 220V y 50Hz. Determine:
a)	la corriente por el condensador.
b)	la corriente por la inductancia.
c)	la corriente total por la fuente.
d)	la potencia total disipada.
Construir el diagrama vectorial en el plano complejo para cada paso.



Problema 2:
Una resistencia R, un condensador C y una inductancia L están conectados en serie.
a)	Calcular la impedancia compleja de la combinación y su valor en resonancia (esto es, cuando la reactancia X se anula).
b)	Construir el diagrama vectorial. Empleándolo, hallar el valor de la impedancia cuando X = R y para la resonancia. Notar que existen dos valores de frecuencia (ω2 y ω1) para los cuales se tiene X = R.
c)	Trazar la curva de resonancia y hallar el ancho de banda (ω2 - ω1).
d)	Repetir los puntos anteriores suponiendo ahora que los mismos componentes se conectan
en paralelo.



Problema 3:
Tres impedancias Z1, Z2, y Z3  están conectadas en paralelo a una fuente de 40V y 50Hz. Suponiendo que  Z1 = 10Ω, Z2 = 20 (1+j) Ω y Z3 = (3−4j) Ω:
a)	calcular la admitancia, conductancia y susceptancia en cada rama.
b)	calcular la conductancia y la susceptancia resultante de la combinación.
c)	calcular la corriente en cada rama, la corriente resultante y la potencia total disipada.
d)	trazar el diagrama vectorial del circuito.



Problema 4:
Una inductancia L que tiene una resistencia interna r está conectada en serie con otra resistencia R = 200 Ω. Cuando estos elementos están conectados a una fuente de 220 V y 50Hz, la caída de tensión sobre la resistencia R es de 50V. Si se altera solamente la frecuencia de la fuente, de modo que sea 60 Hz, la tensión sobre R pasa a ser 44 V. Determinar los valores de L y r.



Problema 5:
En el circuito indicado, la fuente de tensión E entrega 100V con una frecuencia de 50Hz y los elementos que lo constituyen son:
C = 20 µF, L = 0.25 H, y R1 = R2 = R3 = 10 Ω.
a)	Calcular la impedancia equivalente a la derecha de los puntos A y B
b)	Calcular la corriente que circula por cada resistencia.
c)	Construir el diagrama vectorial del circuito.
 




Problema 6:
Para el circuito de la figura:
a)	hallar el valor de la impedancia compleja equivalente.
b)	determinar su valor en resonancia.
c)	¿cuánto vale la frecuencia ω en este caso?
d)	construir el diagrama vectorial de la corriente por cada una
de las ramas.



Problema 7: (Optativo)
Para el circuito de la figura, hallar:
a)	las	corrientes	que	circulan	por	cada	rama empleando el método de mallas.
b)	la potencia suministrada por cada generador.
c)	la potencia disipada en cada impedancia.
Datos: V1  = 30V, V2  = 20 V, Z1  = 5 Ω, Z2  = 4 Ω,  Z3  = (2+3j) Ω, Z4  = 5j Ω, Z5  = 6  Ω y f = 50Hz.

