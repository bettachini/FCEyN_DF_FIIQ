
Guía 2: Conductores, Capacidad, Condensadores, Dieléctricos, Polarización, Campos E y D.

FALTO ESTE PROBLEMA DE LA GUIA 2

 
Problema 3:
En un campo eléctrico uniforme cargado con carga total Q.

E0    se introduce un cuerpo conductor de forma arbitraria
 
a)	¿Qué valor tiene la fuerza eléctrica que se ejerce sobre el cuerpo?
b)	Como consecuencia de la inducción de cargas sobre la superficie del conductor, el campo dejará de ser uniforme en la vecindad del cuerpo. Si se "congela'' la distribución superficial de carga y se quita el campo externo, ¿cómo será el campo en el interior del cuerpo? Notar que al congelar la carga superficial, el cuerpo pierde las propiedades de un conductor.


---------------------------------------------------------------


Guía 6: Circuitos de Corriente Alterna

Problema 1:
Un condensador C = 1µF está conectado en paralelo con una inductancia L = 0.1 H cuya resistencia interna vale R = 1Ω. Se conecta la combinación a una fuente alterna de 220V y 50Hz. Determine:
a)	la corriente por el condensador.
b)	la corriente por la inductancia.
c)	la corriente total por la fuente.
d)	la potencia total disipada.
Construir el diagrama vectorial en el plano complejo para cada paso.



Problema 2:
Una resistencia R, un condensador C y una inductancia L están conectados en serie.
a)	Calcular la impedancia compleja de la combinación y su valor en resonancia (esto es, cuando la reactancia X se anula).
b)	Construir el diagrama vectorial. Empleándolo, hallar el valor de la impedancia cuando X = R y para la resonancia. Notar que existen dos valores de frecuencia (ω2 y ω1) para los cuales se tiene X = R.
c)	Trazar la curva de resonancia y hallar el ancho de banda (ω2 - ω1).
d)	Repetir los puntos anteriores suponiendo ahora que los mismos componentes se conectan
en paralelo.



Problema 3:
Tres impedancias Z1, Z2, y Z3  están conectadas en paralelo a una fuente de 40V y 50Hz. Suponiendo que  Z1 = 10Ω, Z2 = 20 (1+j) Ω y Z3 = (3−4j) Ω:
a)	calcular la admitancia, conductancia y susceptancia en cada rama.
b)	calcular la conductancia y la susceptancia resultante de la combinación.
c)	calcular la corriente en cada rama, la corriente resultante y la potencia total disipada.
d)	trazar el diagrama vectorial del circuito.



Problema 4:
Una inductancia L que tiene una resistencia interna r está conectada en serie con otra resistencia R = 200 Ω. Cuando estos elementos están conectados a una fuente de 220 V y 50Hz, la caída de tensión sobre la resistencia R es de 50V. Si se altera solamente la frecuencia de la fuente, de modo que sea 60 Hz, la tensión sobre R pasa a ser 44 V. Determinar los valores de L y r.



Problema 5:
En el circuito indicado, la fuente de tensión E entrega 100V con una frecuencia de 50Hz y los elementos que lo constituyen son:
C = 20 µF, L = 0.25 H, y R1 = R2 = R3 = 10 Ω.
a)	Calcular la impedancia equivalente a la derecha de los puntos A y B
b)	Calcular la corriente que circula por cada resistencia.
c)	Construir el diagrama vectorial del circuito.
 




Problema 6:
Para el circuito de la figura:
a)	hallar el valor de la impedancia compleja equivalente.
b)	determinar su valor en resonancia.
c)	¿cuánto vale la frecuencia ω en este caso?
d)	construir el diagrama vectorial de la corriente por cada una
de las ramas.



Problema 7: (Optativo)
Para el circuito de la figura, hallar:
a)	las	corrientes	que	circulan	por	cada	rama empleando el método de mallas.
b)	la potencia suministrada por cada generador.
c)	la potencia disipada en cada impedancia.
Datos: V1  = 30V, V2  = 20 V, Z1  = 5 Ω, Z2  = 4 Ω,  Z3  = (2+3j) Ω, Z4  = 5j Ω, Z5  = 6  Ω y f = 50Hz.

