\documentclass[problemas]{guia}

\def \practnum {3} 
\def \practica {Corrientes estacionarias en circuitos pasivos}

\def \materia {Laboratorio de F\'\i sica II para Qu\'\i micos}
\def \periodo {2do. Cuatrimestre de 2015}
\def \catedra {Pablo Cobelli}
\def \website {http://materias.df.uba.ar/f2qa2015c2}
 
\usepackage{graphics}
\usepackage{amsmath}
\usepackage{amsfonts}
\usepackage{graphicx}
\usepackage{float}
\usepackage{wrapfig}
\usepackage{subfigure}
\usepackage{bm}
\usepackage{grffile}
\usepackage{color}
\usepackage{framed}
\usepackage[utf8]{inputenc}
\usepackage[T1]{fontenc}
\usepackage{lmodern}
\usepackage{circuitikz}
\usepackage[spanish]{babel}
\usepackage{babelbib}
\selectbiblanguage{spanish}

 
\usepackage{enumerate}

%----------------------------------------------------------
% Agrega al path de figuras el subdirectorio con el mismo
%     nombre que el archivo principal del proyecto
\graphicspath{{./\jobname/}}

%----------------------------------------------------------
% Definicion del entorno 'sabermas'
\makeatletter
\definecolor{shadecolor}{rgb}{0.89,0.91,0.94}
\newenvironment{sabermas}[1]{%
\vfill
\begin{shaded}
  \begin{center}
  {\textsection{Para saber m\'as}}
  \end{center}
  #1
\sf } 
{%
\end{shaded}%
}
\makeatother

%----------------------------------------------------------
% Definicion del entorno 'problema'
\newcounter{ContadorProblema}
\setcounter{ContadorProblema}{0}
\newcounter{TieneFiguraAsociada}
\setcounter{TieneFiguraAsociada}{0}
\newcounter{UbicacionFigura}
\setcounter{UbicacionFigura}{0}

\newenvironment{problema}[2][]
{%
    \ifx\relax#1\relax%
        \setcounter{TieneFiguraAsociada}{0}
        \else
        \setcounter{TieneFiguraAsociada}{1}
    \fi
    \def \archivofigura {#1}
    % 
    \refstepcounter{ContadorProblema}
    \noindent%
    \ifnum\value{TieneFiguraAsociada} < 1%
        {\sffamily \bfseries Problema \arabic{ContadorProblema}.}
        %{\sc {#1}}%
        \par\nobreak\par\nobreak%
        \medskip 
    \else
        % Va con figura; resta determinar de que lado.
        \ifnum\value{UbicacionFigura} < 1
            % Poner la figura del lado derecho
            \begin{minipage}{12.25cm}
            {\sffamily \bfseries Problema \arabic{ContadorProblema}.}
            %{\sc {#1}}%
            \par\nobreak\par\nobreak%
            \medskip 
        \else
            % Poner la figura del lado izquierdo
            \begin{minipage}{4.5cm}
                \centering
                \includegraphics[width=4.5cm]{\archivofigura}
                {\footnotesize {\sffamily Esquema asociado al 
                problema \arabic{ContadorProblema}}.}
            \end{minipage}\hfill%
            \begin{minipage}{12.25cm}
                {\sffamily \bfseries Problema \arabic{ContadorProblema}.}
                %{\sc {#1}}%
                \par\nobreak\par\nobreak%
                \medskip 
        \fi
    \fi
}
{%
    \ifnum\value{TieneFiguraAsociada} < 1%
        % \par \bigskip \vskip 0.3cm
    \else
        % Va con figura; resta determinar de que lado.
        \ifnum\value{UbicacionFigura} < 1
            % Poner la figura del lado derecho
            \end{minipage}\hfill%
            \begin{minipage}{4.5cm}
                \centering
                \includegraphics[width=4.5cm]{\archivofigura}
                {\footnotesize {\sffamily Esquema asociado al 
                problema \arabic{ContadorProblema}}.}
            \end{minipage}
        \else
            % Poner la figura del lado izquierdo
            \end{minipage}%
        \fi
    \fi
    \setcounter{TieneFiguraAsociada}{0}
    \par \bigskip \vskip 0.3cm
    % Permutamos el valor de la ubicacion
    \ifnum\value{UbicacionFigura} < 1
        \setcounter{UbicacionFigura}{1}
    \else
        \setcounter{UbicacionFigura}{0}
    \fi
}

%----------------------------------------------------------
% Definicion/Redefinicion de estilos
\renewcommand{\vec}[1]{\ensuremath{\mathbf{#1}}}



\hyphenation{ coe-fi-cien-tes coe-fi-cien-te au-to-va-lor
              au-to-va-lo-res co-rres-pon-der pro-ble-ma 
              cual-quie-ra po-la-ri-za-cio-nes }

% \graphicspath{{./guia8/}}

\begin{document} 
\maketitle

\begin{problema}{}
    Un cable de cobre (resistividad del Cu: $1.7\times10^{-8}$ $\Omega$ m) de 
    2 mm de radio y 1 m de longitud se estira hasta cuadruplicar su longitud 
    (las secciones inicial y final son uniformes).
    \begin{enumerate}[(a)]
        \item Calcular la resistencia antes y después del estiramiento, 
            suponiendo que la resistividad no varía.
        \item Por el cable de cobre de 2 mm$^2$ de sección circula una 
            corriente de 1 A. Si hay un electrón de conducción por cada átomo,
            encuentre la velocidad media de los electrones. \\
            Datos: $\delta_{Cu} = 9$ g/cm$^3$, $e = 1.60\times10^{-19}$ C, 
            $N_A = 6\times10^{23}$/mol, $A_{Cu} = 63.5$. 
        \item Calcular  la  resistencia  eléctrica  de  una  plancha,  una  
            estufa  de  cuarzo,  una  lamparita eléctrica de 60 W y una 
            lamparita de linterna.
    \end{enumerate}
\end{problema}

\begin{problema}{}
    Para el circuito representado en la figura de la derecha:
    \begin{enumerate}[(a)]
        \item Calcular las corrientes de ramas y de mallas.
        \item Repetir después de cambiar una de las resistencias de 12 
            $\Omega$ por una de 6 $\Omega$.
        \item Calcular la potencia disipada por cada resistencia y la
            entregada por  la  fuente  en  los  puntos  anteriores.  Verificar
            que  la  condición  para  la  máxima transferencia de potencia se
            cumple.
        \item Calcular el consumo en kWh luego de dos días de funcionamiento 
            en los dos casos.
    \end{enumerate}
\end{problema}

\begin{problema}{}
    Para el circuito que muestra la figura de la izquierda, calcular:
    \begin{enumerate}[(a)]
        \item las corrientes $i_1$ e $i_2$.
        \item la diferencia de potencial entre los puntos C y D.
        \item la potencia disipada por las resistencias de 5 $\Omega$.
        \item Se coloca un amperímetro en serie con la batería de 20 V.	¿Qué 
            corriente mide si la resistencia interna del amperímetro es 
            $Ra = 1$ $\Omega$?
        \item Repita el punto anterior pero ahora considerando que el 
            amperímetro está en serie con la resistencia de 3 $\Omega$.
        \item Comparar los dos puntos anteriores con el primero.
    \end{enumerate}
\end{problema}

\begin{problema}{}
    En el circuito de la figura calcular:
    \begin{enumerate}[(a)]
        \item la resistencia equivalente vista desde la fuente.
        \item la corriente $i$ y la caída de potencial entre los puntos B y C.
        \item la potencia entregada por la fuente.
    \end{enumerate}
\end{problema}

\begin{problema}{}
    Determinar  la  potencia  suministrada  a  una  resistencia  que  se 
    conecta entre A y B si su valor es:
    \begin{enumerate}[(a)]
        \item $R_1 = 1$ $\Omega$.
        \item $R_2 = 5$ $\Omega$.
        \item $R_3 = 10$ $\Omega$.
        \item $R_4$ tal que la transferencia de potencia resulte máxima.
    \end{enumerate}
\end{problema}

\begin{problema}{}
    \begin{enumerate}[(a)]
        \item Obtener el circuito equivalente de Thevenin para el puente de la
            figura (conocido como puente de Wheatstone) visto desde los puntos 
            A y B.
        \item Entre A y B se conecta un galvanómetro de resistencia interna R.
            Calcular la corriente que circula por él en función de 
            $\varepsilon, R_1, R_2, R_3, R_4$ y $R$.
        \item Determine  la  relación  entre  las  resistencias  para  la cual
            la corriente que circula por el amperímetro es nula. Ésta se llama
            condición de equilibrio del puente y se emplea para medir 
            resistencias con precisión.
        \item Hallar  la  potencia  disipada  por  el  galvanómetro  cuando:
            $\varepsilon = 1$ V, $R_4 = 1.1$ $\Omega$, $R_1 = R_2 = R_3 = 1$ 
            $\Omega$  y $R = 0.1$ $\Omega$.
    \end{enumerate}
\end{problema}

\end{document}
