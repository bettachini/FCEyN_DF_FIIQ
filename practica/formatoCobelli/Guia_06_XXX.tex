\documentclass[problemas]{guia}

\def \practnum {6} 
\def \practica {Circuitos de corriente alterna}

\def \materia {Laboratorio de Fisica II para Quimicos}
\def \periodo {2do. Cuatrimestre de 2015}
\def \catedra {P. Cobelli}
 
% \usepackage{graphics}
% \usepackage{amsmath}
\usepackage{amsfonts}
\usepackage{graphicx}
\usepackage[arrowdel]{physics}
\usepackage{float}
\usepackage{wrapfig}
\usepackage{subfigure}
% \usepackage{bm}
% \usepackage{grffile}
\usepackage{color}
% \usepackage{framed}
\usepackage[utf8]{inputenc}
% \usepackage[T1]{fontenc}
% \usepackage{lmodern}

\usepackage[separate-uncertainty=true, multi-part-units=single, locale=FR]{siunitx}

\usepackage{tikz}
\usepackage[siunitx]{circuitikz}

% circuitikz: rotate instruments - https://tex.stackexchange.com/questions/105864/rotate-voltmeter-circuitikz
\newcommand{\mymeter}[2] 
{  % #1 = name , #2 = rotation angle
\begin{scope}[transform shape,rotate=#2]
\draw[thick] (#1)node(){$\mathbf V$} circle (11pt);
\draw[rotate=45,-latex] (#1)  +(-17pt,0) --+(17pt,0);
\end{scope}
}

\usepackage{isotope} % $\isotope[A][Z]{X}\to\isotope[A-4][Z-2]{Y}+\isotope[4][2]{\alpha}$

\usepackage[spanish]{babel}
\usepackage{babelbib}
\selectbiblanguage{spanish}
 
\usepackage{enumerate}
% definicion del entorno 'sabermas'
\makeatletter
\definecolor{shadecolor}{rgb}{0.89,0.91,0.94}
\newenvironment{sabermas}[1]{%
\vfill
\begin{shaded}
  \begin{center}
  {\textsection{Para saber m\'as}}
  \end{center}
  #1
\sf } 
{%
\end{shaded}%
}
\makeatother

\renewcommand{\vec}[1]{\ensuremath{\mathbf{#1}}}



\input{hyphenation_rules}
% \graphicspath{{./guia8/}}

\begin{document} 
\maketitle

\begin{problema}{}
    Un condensador $C = 1$ $\mu$F está conectado en paralelo con una 
    inductancia $L = 0.1$ H cuya resistencia interna vale $R = 1$ $\Omega$. 
    Se conecta la combinación a una fuente alterna de 220 V y 50 Hz. Determine:
    \begin{enumerate}[(a)]
        \item la corriente por el condensador,
        \item la corriente por la inductancia,
        \item la corriente total por la fuente,
        \item la potencia total disipada.
\end{enumerate}
    Construir el diagrama vectorial en el plano complejo para cada paso.
\end{problema}

\begin{problema}{}
    Una resistencia $R$, un condensador $C$ y una inductancia $L$ están 
    conectados en serie.
    \begin{enumerate}[(a)]
        \item Calcular la impedancia compleja de la combinación y su valor en 
            resonancia (esto es, cuando la reactancia $X$ se anula).
        \item Construir el diagrama vectorial. Empleándolo, hallar el valor de 
            la impedancia cuando $X = R$ y para la resonancia. Notar que 
            existen dos valores de frecuencia ($\omega_1$ y $\omega_2$) para 
            los cuales se tiene $X = R$.
        \item Trazar la curva de resonancia y hallar el ancho de banda 
            $(\omega_2 - \omega_1)$.
        \item Repetir los puntos anteriores suponiendo ahora que los mismos 
            componentes se conectan en paralelo.
    \end{enumerate}
\end{problema}

\begin{problema}{}
    Tres impedancias $Z_1, Z_2$ y $Z_3$ están conectadas en paralelo a una 
    fuente de 40 V y 50 Hz. Suponiendo que $Z_1 = 10$ $\Omega$, 
    $Z_2 = 20 \: (1+j)$ $\Omega$ y $Z_3 = (3-4j)$ $\Omega$:
    \begin{enumerate}[(a)]
        \item Calcular la admitancia, conductancia y susceptancia en cada rama.
        \item Calcular la conductancia y la susceptancia resultante de la 
            combinación.
        \item Calcular la corriente en cada rama, la corriente resultante y la
            potencia total disipada.
        \item Trazar el diagrama vectorial del circuito.
    \end{enumerate}
\end{problema}

\begin{problema}{}
    Una inductancia $L$ que tiene una resistencia interna $r$ está conectada en
    serie con otra resistencia $R = 200$ $\Omega$. Cuando estos elementos están
    conectados a una fuente de 220 V y 50 Hz, la caída de tensión sobre la 
    resistencia $R$ es de 50 V. Si se altera solamente la frecuencia de la 
    fuente, de modo que sea 60 Hz, la tensión sobre $R$ pasa a ser 44 V. 
    Determinar los valores de $L$ y $r$.
\end{problema}


\begin{problema}{}
    En el circuito indicado, la fuente de tensión $E$ entrega 100 V con una 
    frecuencia de 50 Hz y los elementos que lo constituyen son:
    $C = 20$ $\mu$F, $L = 0.25$ H, y $R_1 = R_2 = R_3 = 10$ $\Omega$. 
    \begin{enumerate}[(a)]
        \item Calcular la impedancia equivalente a la derecha de los puntos 
            A y B.
        \item Calcular la corriente que circula por cada resistencia.
        \item Construir el diagrama vectorial del circuito.
\end{enumerate}
\end{problema}

\begin{problema}{}
    Para el circuito de la figura:
    \begin{enumerate}[(a)]
        \item Hallar el valor de la impedancia compleja equivalente.
        \item Determinar su valor en resonancia.
        \item ¿Cuánto vale la frecuencia $\omega$ en este caso?
        \item Construir el diagrama vectorial de la corriente por cada una
            de las ramas.
    \end{enumerate}
\end{problema}

\begin{problema}{}
    Para el circuito de la figura, hallar:
    \begin{enumerate}[(a)]
        \item Las	corrientes	que	circulan por	cada	rama empleando el 
            método de mallas.
        \item La potencia suministrada por cada generador.
        \item La potencia disipada en cada impedancia. \\

    Datos: $V_1 = 30$ V, $V_2 = 20$ V, $Z_1 = 5$ $\Omega$, $Z_2 = 4$ $\Omega$,
    $Z_3  = (2+3j)$ $\Omega$, $Z_4 = 5j$ $\Omega$, $Z_5 = 6$ $\Omega$ y 
    $f = 50$ Hz.
    \end{enumerate}
\end{problema}

\end{document}
