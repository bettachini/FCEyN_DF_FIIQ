\documentclass[11pt,spanish,a4paper]{article}
% Versión 2.o cuat 2015 Víctor Bettachini < bettachini@df.uba.ar >

\usepackage{babel}
\addto\shorthandsspanish{\spanishdeactivate{~<>}}
\usepackage[utf8]{inputenc}
\usepackage{float}

\usepackage{units}
\usepackage[separate-uncertainty=true, multi-part-units=single, locale=FR]{siunitx}

\usepackage{amsmath}
\usepackage{amstext}
\usepackage{amssymb}

\newcommand{\pvec}[1]{\vec{#1}\mkern2mu\vphantom{#1}}

% \usepackage{tikz}
% \input{DimLinesTikz}
% \usetikzlibrary{decorations.pathmorphing, patterns}

\usepackage{graphicx}
\graphicspath{{./graphs/}}

\usepackage[margin=1.3cm,nohead]{geometry}
% \voffset-3.5cm
% \hoffset-3cm
% \setlength{\textwidth}{17.5cm}
% \setlength{\textheight}{27cm}

\usepackage{lastpage}
\usepackage{fancyhdr}
\pagestyle{fancyplain}
\fancyhead{}
\fancyfoot{{\tiny \textcopyright Departamento de Física, FCEyN, UBA}}
\fancyfoot[C]{ {\tiny Actualizado al \today} }
\fancyfoot[RO, LE]{Pág. \thepage/\pageref{LastPage}}
\renewcommand{\headrulewidth}{0pt}
\renewcommand{\footrulewidth}{0pt}

% lista multicolumna
\usepackage{multicol}

% textsubscript
\usepackage{changes}

% \def \materia {Física II para químicos}
\def \periodo {cuatrimestre de verano - 2017}
\def \website {http://materias.df.uba.ar/f2qa2017v}


\begin{document}
\noindent
\textbf{Física II (Químicos)}\hfill \textcopyright {\tt DF, FCEyN, UBA}
% \textbf{\materia}\hfill \periodo
\begin{center}
  \textsc{\large Guía 7: Ondas unidimensionales - Viajeras y estacionarias - En gases: sonido}
\par\end{center}{\large \par}


\begin{enumerate}

\section*{Viajeras y estacionarias}

%\item Determinar cuáles de las siguientes expresiones matemáticas pueden representar ondas viajeras unidimensionales, físicamente razonables.
%	\begin{multicols}{2}
%	\begin{enumerate}
%		\item \( \varphi(x,t)= A \mathrm{e}^{- \alpha (x- c t) } \)
%		\item \( \varphi(x,t)= A \mathrm{e}^{- 2 \alpha (x- c t) } \)
%		\item \( \varphi(x,t)= A \ln \left( k (x- c t) \right) \)
%		\item \( \varphi(x,t)= A \left( x- c t \right) \)
%		\item \( \varphi(x,t)= A \left( x- c t \right)^n \)
%		\item \( \varphi(x,t)= A \sin \left( k (x- c t) \right) \)
%		\item \( \varphi(x,t)= A \sin \left( \alpha (x^2- c^2 t^2) \right) \)
%		\item \( \varphi(x,t)= A \left( x+ c t \right)^{1/2} \)
%	\end{enumerate}
%	\end{multicols}


\item
Sean dos ondas que se superponen entre sí:
	\[
	\psi_1(x,t)= A_1 \sin \left( \omega t - k x + \varphi_1 \right) \qquad \mathrm{y} \qquad \psi_1(x,t)= A_2 \sin \left( \omega t - k x + \varphi_2 \right), 
	\]
en las que \( \epsilon_1 \) y \( \epsilon_2 \) son independientes del tiempo.
    \begin{enumerate}
        \item Determine la perturbación resultante.
		\item Hágalo en particular para los siguientes valores de los parámetros: \( \omega=\SI{120}{\per\second}, A_1=\SI{6}{\milli\metre}, A_2=\SI{8}{\milli\metre}, \varphi_1= 0, \varphi_2= \pi/2, \lambda= \SI{2}{\centi\metre}\).
		\item Grafique cada función de onda y la resultante en función de la posición \(x\) (para \(t=0\)) y en función del tiempo \(t\) (para \(x=0\)).
		\item Si \(A_1= A_2= A\) y se observa que la superposición de \(\psi_1\) y \(\psi_2\) es una onda de amplitud \(A\). ¿Cuanto es la diferencia de fase, \(\varphi_2 - \varphi_1\)?
    \end{enumerate}


\item Encuentre la resultante de superposición de las siguientes dos ondas,
	\[
		\psi_1(x,t)= A \cos \left( k x + \omega t \right) \qquad \mathrm{y} \qquad \psi_2(x,t)= A \cos \left( k x - \omega t \right), 
	\]
	Describa y grafique la onda resultante.
	¿Se obtiene una onda viajera?


\item Sea una onda transversal descripta por:
	\[
	\psi(x,t)= \SI{4}{\milli\metre} \cos \left[ 2 \pi  \left( \frac{t}{\SI{0.05}{\second}} - \frac{x}{\SI{0.25}{\centi\metre}} \right) \right].
	\]
    \begin{enumerate}
        \item Determine su velocidad de propagación, frecuencia, longitud de onda, número de onda y fase inicial.
		\item Considere una partícula del medio en que se transmite la onda ubicada en \(x= 0\) y otra en \(x= \SI{10}{\centi\metre}\). En el instante \(t= 0\), ¿cuál es la diferencia entre las velocidades de oscilación transversal de ambas partículas?
		¿Cuál es la diferencia entre las fases de los movimientos oscilatorios de dichas partículas?
    \end{enumerate}


\item Una cuerda oscila transversalmente de modo que la perturbación está dada por:
	\[
	\psi(x,t)= \SI{0.5}{\centi\metre} \sin \left( \SI{1.26}{\per\centi\metre} x - \SI{12.57}{\per\second} t + \varphi_0 \right) 
	\]
	Se sabe que en el punto \(x= \SI{1.5}{\metre}\) y en el instante \(t= \SI{0.4}{\second}\), la cuerda tiene velocidad negativa y desplazamiento nulo.
	Calcule:
    \begin{enumerate}
		\item la frecuencia de la oscilación,
		\item la longitud de onda,
		\item y la fase inicial \(\varphi_0\).
    \end{enumerate}


\section*{En medios materiales}

\item El extremo de un tubo delgado de goma está fijo a un soporte.
	El otro extremo pasa por una polea situada a \SI{5}{\metre} del extremo fijo y se cuelga de dicho extremo una carga de \SI{2}{\kilo\gram}.
	La masa del tubo entre el extremo fijo y la polea es \SI{0.6}{\kilo\gram}.
	Una onda armónica transversal de \SI{1}{\milli\metre} de amplitud y longitud de onda \SI{30}{\centi\metre} se propaga a lo largo del tubo.
    \begin{enumerate}
		\item Calcule la velocidad de propagación de dicha onda.
		\item Escriba la ecuación que describe la onda.
		\item Calcule la velocidad transversal máxima.
    \end{enumerate}


\item Sea una cuerda de densidad lineal de masa \SI{0.2}{\kilo\gram\per\metre} y longitud \SI{80}{\centi\metre} sometida a una tensión de \SI{80}{\newton}.
    \begin{enumerate}
	\item Calcule la velocidad con que se propagan ondas en esta cuerda.
	\item Un extremo de la cuerda se sujeta a un soporte ideal (o sea un soporte tal que la onda incidente en él se refleja totalmente) y el otro extremo se mueve de modo de generar una onda armónica \(\psi_1 \) que se propaga por la cuerda. 
	Escriba la expresión para las ondas estacionarias que resultan.
	Considere \(\psi_1(x,t)= A \cos \left( k x + \omega t + \frac{\pi}{2} \right) \).
	\item Ambos extremos se sujetan a soportes ideales y se deforma la cuerda de modo de generar ondas estacionarias.
	Encuentre la frecuencia y longitud de onda fundamental y las armónicas.
	Dibuje los primeros tres modos de oscilación de la cuerda.
	\item En las mismas condiciones del punto anterior la cuerda está inicialmente deformada adoptando la forma de su tercer modo normal y con una amplitud de \SI{4.5}{\milli\metre}.
	Calcule la frecuencia de la oscilación y el valor máximo de la velocidad transversal de la cuerda.
	\end{enumerate}



\section*{En gases: sonido}

\item La ecuación de una onda de presión en una columna de gas es:
	\[
		\delta P= A_p \sin{2 \pi \left( \frac{x }{\lambda }- \frac{t }{\tau } \right) }
	\]
	donde \(\delta P\) es la presión medida respecto a la presión del equilibrio.
    \begin{enumerate}
		\item Halle la expresión para las ondas de desplazamiento.
		\item Muestre que las ondas de desplazamiento están desfasadas en \(\pi/2\) respecto de las ondas de presión.
    \end{enumerate}



\item En un tubo cilíndrico cerrado de diámetro \SI{5}{\centi\metre} que contiene aire (\(\rho_a = \SI{1.2}{\kilo\gram\per\metre\cubed}\); \(v_a = \SI{330}{\metre\per\second}\)) la distancia entre dos nodos consecutivos de una onda acústica estacionaria producida en ambos extremos es de \SI{20}{\centi\metre}.
	Determine:
    \begin{enumerate}
		\item la frecuencia de la onda sonora,
		\item la amplitud máxima de la onda de presión si la amplitud máxima de la onda de desplazamiento es de \SI{10}{\micro\metre},
		\item y la potencia de la onda sonora (\(\mathrm{Pot}= S v \delta p \), \(S\) la sección, \(v\) la velocidad de propagación, y \(\delta p\) la presión por sobre la atmosférica).
	\end{enumerate}


\item Explique por qué se oye la vibración de un diapasón.
	¿Cuánto valen las frecuencias límites que estimulan al oído humano?
	¿Por qué es conveniente adosar el diapasón a una caja de resonancia?


\item \begin{enumerate}
		\item Una cuerda de violín de \SI{30}{\centi\metre} de longitud emite la nota La\textsubscript{3} (\SI{440}{\per\second}) en su modo fundamental.
		Calcule las modificaciones que deben realizarse en la longitud para que dé las notas Si\textsubscript{3} (\SI{495}{\per\second}) , Do\textsubscript{3} (\SI{528}{\per\second}) y Re\textsubscript{3} (\SI{594}{\per\second}), todas en su modo fundamental.
		\item Para una dada cuerda (o sea si su longitud, densidad lineal y tensión son fijas), ¿el sonido emitido es de una única frecuencia o es la superposición de armónicos?
		En caso que sea la superposición, ¿a cuál de las frecuencias armónicas corresponde el tono del sonido? 
	\end{enumerate}


\item \begin{enumerate}
    	\item \label{13a} ¿Cuánto vale la menor longitud que puede tener un tubo de órgano abierto en ambos extremos para que produzca en el aire un sonido de \SI{440}{\hertz}?
		\item Para producir el mismo tono que el primer armónico de este tubo, ¿qué longitud debería tener un tubo cerrado?
		\item Y para producir la misma nota que su fundamental, ¿qué tensión debierá aplicarse a la cuerda de un violín que tiene \SI{50}{\centi\metre} de longitud y una masa de \SI{2}{g}?
		\item Para esta cuerda calcule la longitud de onda de la oscilación, 
		\item y la del sonido producido.
	\end{enumerate}

\item El nivel de agua en una probeta de \SI{1}{\metre} de longitud puede ser ajustado a voluntad.
	Se coloca un diapasón sobre el extremo abierto del tubo.
	El mismo oscila en una frecuencia de \SI{600}{\hertz}.
	¿Para qué niveles de agua habrá resonancia?



\end{enumerate}
\end{document}
