\documentclass[problemas]{guia}

\def \practnum {4} 
\def \practica {Magnetost\'atica y Ley de Amp\`ere}

\def \materia {Laboratorio de F\'\i sica II para Qu\'\i micos}
\def \periodo {2do. Cuatrimestre de 2015}
\def \catedra {Pablo Cobelli}
\def \website {http://materias.df.uba.ar/f2qa2015c2}
 
\usepackage{graphics}
\usepackage{amsmath}
\usepackage{amsfonts}
\usepackage{graphicx}
\usepackage{float}
\usepackage{wrapfig}
\usepackage{subfigure}
\usepackage{bm}
\usepackage{grffile}
\usepackage{color}
\usepackage{framed}
\usepackage[utf8]{inputenc}
\usepackage[T1]{fontenc}
\usepackage{lmodern}
\usepackage{circuitikz}
\usepackage[spanish]{babel}
\usepackage{babelbib}
\selectbiblanguage{spanish}

 
\usepackage{enumerate}

%----------------------------------------------------------
% Agrega al path de figuras el subdirectorio con el mismo
%     nombre que el archivo principal del proyecto
\graphicspath{{./\jobname/}}

%----------------------------------------------------------
% Definicion del entorno 'sabermas'
\makeatletter
\definecolor{shadecolor}{rgb}{0.89,0.91,0.94}
\newenvironment{sabermas}[1]{%
\vfill
\begin{shaded}
  \begin{center}
  {\textsection{Para saber m\'as}}
  \end{center}
  #1
\sf } 
{%
\end{shaded}%
}
\makeatother

%----------------------------------------------------------
% Definicion del entorno 'problema'
\newcounter{ContadorProblema}
\setcounter{ContadorProblema}{0}
\newcounter{TieneFiguraAsociada}
\setcounter{TieneFiguraAsociada}{0}
\newcounter{UbicacionFigura}
\setcounter{UbicacionFigura}{0}

\newenvironment{problema}[2][]
{%
    \ifx\relax#1\relax%
        \setcounter{TieneFiguraAsociada}{0}
        \else
        \setcounter{TieneFiguraAsociada}{1}
    \fi
    \def \archivofigura {#1}
    % 
    \refstepcounter{ContadorProblema}
    \noindent%
    \ifnum\value{TieneFiguraAsociada} < 1%
        {\sffamily \bfseries Problema \arabic{ContadorProblema}.}
        %{\sc {#1}}%
        \par\nobreak\par\nobreak%
        \medskip 
    \else
        % Va con figura; resta determinar de que lado.
        \ifnum\value{UbicacionFigura} < 1
            % Poner la figura del lado derecho
            \begin{minipage}{12.25cm}
            {\sffamily \bfseries Problema \arabic{ContadorProblema}.}
            %{\sc {#1}}%
            \par\nobreak\par\nobreak%
            \medskip 
        \else
            % Poner la figura del lado izquierdo
            \begin{minipage}{4.5cm}
                \centering
                \includegraphics[width=4.5cm]{\archivofigura}
                {\footnotesize {\sffamily Esquema asociado al 
                problema \arabic{ContadorProblema}}.}
            \end{minipage}\hfill%
            \begin{minipage}{12.25cm}
                {\sffamily \bfseries Problema \arabic{ContadorProblema}.}
                %{\sc {#1}}%
                \par\nobreak\par\nobreak%
                \medskip 
        \fi
    \fi
}
{%
    \ifnum\value{TieneFiguraAsociada} < 1%
        % \par \bigskip \vskip 0.3cm
    \else
        % Va con figura; resta determinar de que lado.
        \ifnum\value{UbicacionFigura} < 1
            % Poner la figura del lado derecho
            \end{minipage}\hfill%
            \begin{minipage}{4.5cm}
                \centering
                \includegraphics[width=4.5cm]{\archivofigura}
                {\footnotesize {\sffamily Esquema asociado al 
                problema \arabic{ContadorProblema}}.}
            \end{minipage}
        \else
            % Poner la figura del lado izquierdo
            \end{minipage}%
        \fi
    \fi
    \setcounter{TieneFiguraAsociada}{0}
    \par \bigskip \vskip 0.3cm
    % Permutamos el valor de la ubicacion
    \ifnum\value{UbicacionFigura} < 1
        \setcounter{UbicacionFigura}{1}
    \else
        \setcounter{UbicacionFigura}{0}
    \fi
}

%----------------------------------------------------------
% Definicion/Redefinicion de estilos
\renewcommand{\vec}[1]{\ensuremath{\mathbf{#1}}}



\hyphenation{ coe-fi-cien-tes coe-fi-cien-te au-to-va-lor
              au-to-va-lo-res co-rres-pon-der pro-ble-ma 
              cual-quie-ra po-la-ri-za-cio-nes }

% \graphicspath{{./guia8/}}

\begin{document} 
\maketitle


\begin{problema}{}
    Una partícula de carga $q$ se mueve en un campo magnético uniforme 
    $\vec{B}$ con una velocidad $v$ perpendicular al campo.
    \begin{enumerate}[(a)]
        \item Calcule el radio de la órbita circular descripta. ¿Aumenta el 
            módulo de la velocidad? ¿Por qué?
        \item Determine la frecuencia del movimiento circular descripto.
        \item ¿Qué sucede si la velocidad es paralela al campo magnético? ¿Y 
            si tiene una componente paralela al campo y otra perpendicular?
    \end{enumerate}
\end{problema}

\begin{problema}{}
    Una partícula de carga $q$ entra, con una velocidad $v$, en una región del
    espacio donde existe un campo eléctrico uniforme $\vec{E}$ de 80 kV/m 
    dirigido hacia abajo, como se muestra en la figura. Perpendicular al campo 
    eléctrico y a la velocidad de la partícula cargada, se halla un campo
    magnético $\vec{B}$ de $0.4$ T. Si la rapidez de la partícula se escoge 
    apropiadamente, ésta no sufrirá ninguna deflexión a causa de los campos 
    perpendiculares. ¿Qué rapidez debe ser seleccionada en este caso? 
    (Este dispositivo se llama selector de velocidades).
\end{problema}

\begin{problema}{}
    La figura muestra un dispositivo empleado para la medición de la masa de 
    los iones. Un ion de masa $m$ y carga $+q$ sale esencialmente en reposo 
    de la fuente S, cámara donde se produce la descarga de un gas. La 
    diferencia de potencial $V$ acelera el ion y se permite que entre  en  una 
    región  con  un  campo  magnético  perpendicular uniforme $\vec{B}$. 
    Dentro del campo, el ion se mueve en semicírculo, chocando con una placa 
    fotográfica a la distancia $x$ de la rendija de entrada. Demuestre que la 
    masa $m$ del ion está dada por:
    \begin{equation*}
        ECUACION
    \end{equation*}
\end{problema}

\begin{problema}{}
    Calcule la fuerza por unidad de longitud entre dos cables paralelos por los
    que circula una corriente de 30 A. La separación entre cables es de 2 cm. 
    Estime hasta qué distancia por encima de los cables se verá afectada la 
    indicación de una brújula. Considere los dos posibles sentidos de 
    circulación de la corriente. (Suponga que la intensidad del campo magnético
    terrestre en el lugar es de $5\times10^{-4}$ T y forma un ángulo de 
    30$^\circ$ con la vertical).
\end{problema}

\begin{problema}{}
    \begin{enumerate}[(a)]
        \item Calcule el campo magnético sobre el eje de una espira circular 
            de área $A$ y corriente $I$.
        \item Repita el cálculo para una espira cuadrada.
        \item Estudie  y compare  los  comportamientos  de  ambos  resultados
            para  distancias  grandes. Expréselos en función de los momentos 
            magnéticos de las espiras.
    \end{enumerate}
\end{problema}

\begin{problema}{}
    \begin{enumerate}[(a)]
        \item Calcule el campo magnético sobre el eje de un solenoide de 
            longitud $L$, con $N$ vueltas devanadas densamente, por el que 
            circula una corriente $I$.
        \item Estudie  el  comportamiento  a  grandes  distancias  y  encuentre
            el  valor  del  momento magnético del solenoide.
        \item Obtenga el límite de solenoide infinito.
        \item Suponga que el solenoide tiene 40 cm de largo, 10 cm de diámetro 
            y el campo en el centro es de 3 T (éste es un campo muy intenso). 
            Si el solenoide se encuentra en el subsuelo del pabellón I, 
            ¿influirá en la medición del campo magnético terrestre que 
            realizan los alumnos en el segundo piso?
    \end{enumerate}
\end{problema}

\begin{problema}{}
    Calcule la fuerza sobre una aguja pequeña magnetizada con momento 
    magnético $\vec{m}$, colocada sobre el eje del solenoide finito del 
    problema anterior. Exprese la fuerza en función de la distancia al centro 
    del solenoide. Discuta el sentido de la fuerza en relación a los sentidos 
    del momento magnético $\vec{m}$ y el campo magnético $\vec{B}$.
\end{problema}

\begin{problema}{}
    Dibuje cualitativamente las líneas de campo magnético correspondientes a 
    dos cables rectilíneos infinitos y paralelos, que conducen sendas 
    corrientes $I$ de sentido contrario. Tenga en cuenta cuál debe ser el 
    comportamiento del campo cerca y lejos de los cables.
\end{problema}

\begin{problema}{}
    Aprovechando la simetría de la distribución de corrientes y usando la 
    ley de Ampère, determine el vector campo magnético en los siguientes casos:
    \begin{enumerate}[(a)]
        \item un cable rectilíneo infinito por el que circula una 
            corriente $I$.
        \item un cilindro infinito de radio $R$ por el que circula una densidad
            de corriente uniforme $\vec{j}$.
        \item un solenoide infinito de $n$ vueltas por unidad de longitud y 
            corriente $I$ (suponga que el devanado es suficientemente denso 
            como para despreciar la componente longitudinal de los elementos 
            de corriente).
        \item un plano infinito con densidad superficial de corriente $g$ 
            uniforme.
        \item dos  planos  infinitos  paralelos,  separados  una  distancia  
            $d$,  con  densidades  de  corriente uniformes $g$ y $-g$.
        \item una lámina infinita de caras plano-paralelas y espesor $d$, con 
            densidad de corriente $\vec{j}$ uniforme.
        \item un toroide de radio interior $a$ y radio exterior $b$, con un 
            arrollamiento denso de $N$ vueltas por el que circula una 
            corriente $I$.
    \end{enumerate}
\end{problema}

\begin{problema}{}
    Un cable coaxil está formado por dos conductores cilíndricos coaxiales 
    separados por un medio de permeabilidad $\mu$ (ver figura). Por ambos 
    conductores circulan corrientes $I$ iguales y opuestas. Suponiendo que la 
    densidad  de corriente en  cada uno  de los  conductores  es  uniforme,
    encuentre el campo magnético $\vec{B}$ en todo punto del espacio.
\end{problema}

\begin{problema}{}
    Un cilindro infinito de radio $a$ es circulado por una corriente 
    volumétrica uniforme  $\vec{j} = j_0 \: \hat{z}$, coaxial con el cilindro.
    En la zona $b < r < c$ (con $a < b$), se tiene un medio magnético lineal,
    isótropo y homogéneo cuya permeabilidad relativa es $\mu_r =1000$.
    \begin{enumerate}[(a)]
        \item Calcular los campos $\vec{H}$ y $\vec{B}$ en todo el espacio.
        \item ¿Se comporta el medio material como un blindaje magnético?
    \end{enumerate}
\end{problema}

\end{document}
