\documentclass[problemas]{guia}

\def \practnum {6} 
\def \practica {Circuitos de corriente alterna}

\def \materia {Laboratorio de F\'\i sica II para Qu\'\i micos}
\def \periodo {2do. Cuatrimestre de 2015}
\def \catedra {Pablo Cobelli}
\def \website {http://materias.df.uba.ar/f2qa2015c2}
 
\usepackage{graphics}
\usepackage{amsmath}
\usepackage{amsfonts}
\usepackage{graphicx}
\usepackage{float}
\usepackage{wrapfig}
\usepackage{subfigure}
\usepackage{bm}
\usepackage{grffile}
\usepackage{color}
\usepackage{framed}
\usepackage[utf8]{inputenc}
\usepackage[T1]{fontenc}
\usepackage{lmodern}
\usepackage{circuitikz}
\usepackage[spanish]{babel}
\usepackage{babelbib}
\selectbiblanguage{spanish}

 
\usepackage{enumerate}

%----------------------------------------------------------
% Agrega al path de figuras el subdirectorio con el mismo
%     nombre que el archivo principal del proyecto
\graphicspath{{./\jobname/}}

%----------------------------------------------------------
% Definicion del entorno 'sabermas'
\makeatletter
\definecolor{shadecolor}{rgb}{0.89,0.91,0.94}
\newenvironment{sabermas}[1]{%
\vfill
\begin{shaded}
  \begin{center}
  {\textsection{Para saber m\'as}}
  \end{center}
  #1
\sf } 
{%
\end{shaded}%
}
\makeatother

%----------------------------------------------------------
% Definicion del entorno 'problema'
\newcounter{ContadorProblema}
\setcounter{ContadorProblema}{0}
\newcounter{TieneFiguraAsociada}
\setcounter{TieneFiguraAsociada}{0}
\newcounter{UbicacionFigura}
\setcounter{UbicacionFigura}{0}

\newenvironment{problema}[2][]
{%
    \ifx\relax#1\relax%
        \setcounter{TieneFiguraAsociada}{0}
        \else
        \setcounter{TieneFiguraAsociada}{1}
    \fi
    \def \archivofigura {#1}
    % 
    \refstepcounter{ContadorProblema}
    \noindent%
    \ifnum\value{TieneFiguraAsociada} < 1%
        {\sffamily \bfseries Problema \arabic{ContadorProblema}.}
        %{\sc {#1}}%
        \par\nobreak\par\nobreak%
        \medskip 
    \else
        % Va con figura; resta determinar de que lado.
        \ifnum\value{UbicacionFigura} < 1
            % Poner la figura del lado derecho
            \begin{minipage}{12.25cm}
            {\sffamily \bfseries Problema \arabic{ContadorProblema}.}
            %{\sc {#1}}%
            \par\nobreak\par\nobreak%
            \medskip 
        \else
            % Poner la figura del lado izquierdo
            \begin{minipage}{4.5cm}
                \centering
                \includegraphics[width=4.5cm]{\archivofigura}
                {\footnotesize {\sffamily Esquema asociado al 
                problema \arabic{ContadorProblema}}.}
            \end{minipage}\hfill%
            \begin{minipage}{12.25cm}
                {\sffamily \bfseries Problema \arabic{ContadorProblema}.}
                %{\sc {#1}}%
                \par\nobreak\par\nobreak%
                \medskip 
        \fi
    \fi
}
{%
    \ifnum\value{TieneFiguraAsociada} < 1%
        % \par \bigskip \vskip 0.3cm
    \else
        % Va con figura; resta determinar de que lado.
        \ifnum\value{UbicacionFigura} < 1
            % Poner la figura del lado derecho
            \end{minipage}\hfill%
            \begin{minipage}{4.5cm}
                \centering
                \includegraphics[width=4.5cm]{\archivofigura}
                {\footnotesize {\sffamily Esquema asociado al 
                problema \arabic{ContadorProblema}}.}
            \end{minipage}
        \else
            % Poner la figura del lado izquierdo
            \end{minipage}%
        \fi
    \fi
    \setcounter{TieneFiguraAsociada}{0}
    \par \bigskip \vskip 0.3cm
    % Permutamos el valor de la ubicacion
    \ifnum\value{UbicacionFigura} < 1
        \setcounter{UbicacionFigura}{1}
    \else
        \setcounter{UbicacionFigura}{0}
    \fi
}

%----------------------------------------------------------
% Definicion/Redefinicion de estilos
\renewcommand{\vec}[1]{\ensuremath{\mathbf{#1}}}



\hyphenation{ coe-fi-cien-tes coe-fi-cien-te au-to-va-lor
              au-to-va-lo-res co-rres-pon-der pro-ble-ma 
              cual-quie-ra po-la-ri-za-cio-nes }

% \graphicspath{{./guia8/}}

\begin{document} 
\maketitle

\begin{problema}{}
    Un condensador $C = 1$ $\mu$F está conectado en paralelo con una 
    inductancia $L = 0.1$ H cuya resistencia interna vale $R = 1$ $\Omega$. 
    Se conecta la combinación a una fuente alterna de 220 V y 50 Hz. Determine:
    \begin{enumerate}[(a)]
        \item la corriente por el condensador,
        \item la corriente por la inductancia,
        \item la corriente total por la fuente,
        \item la potencia total disipada.
\end{enumerate}
    Construir el diagrama vectorial en el plano complejo para cada paso.
\end{problema}

\begin{problema}{}
    Una resistencia $R$, un condensador $C$ y una inductancia $L$ están 
    conectados en serie.
    \begin{enumerate}[(a)]
        \item Calcular la impedancia compleja de la combinación y su valor en 
            resonancia (esto es, cuando la reactancia $X$ se anula).
        \item Construir el diagrama vectorial. Empleándolo, hallar el valor de 
            la impedancia cuando $X = R$ y para la resonancia. Notar que 
            existen dos valores de frecuencia ($\omega_1$ y $\omega_2$) para 
            los cuales se tiene $X = R$.
        \item Trazar la curva de resonancia y hallar el ancho de banda 
            $(\omega_2 - \omega_1)$.
        \item Repetir los puntos anteriores suponiendo ahora que los mismos 
            componentes se conectan en paralelo.
    \end{enumerate}
\end{problema}

\begin{problema}{}
    Tres impedancias $Z_1, Z_2$ y $Z_3$ están conectadas en paralelo a una 
    fuente de 40 V y 50 Hz. Suponiendo que $Z_1 = 10$ $\Omega$, 
    $Z_2 = 20 \: (1+j)$ $\Omega$ y $Z_3 = (3-4j)$ $\Omega$:
    \begin{enumerate}[(a)]
        \item Calcular la admitancia, conductancia y susceptancia en cada rama.
        \item Calcular la conductancia y la susceptancia resultante de la 
            combinación.
        \item Calcular la corriente en cada rama, la corriente resultante y la
            potencia total disipada.
        \item Trazar el diagrama vectorial del circuito.
    \end{enumerate}
\end{problema}

\begin{problema}{}
    Una inductancia $L$ que tiene una resistencia interna $r$ está conectada en
    serie con otra resistencia $R = 200$ $\Omega$. Cuando estos elementos están
    conectados a una fuente de 220 V y 50 Hz, la caída de tensión sobre la 
    resistencia $R$ es de 50 V. Si se altera solamente la frecuencia de la 
    fuente, de modo que sea 60 Hz, la tensión sobre $R$ pasa a ser 44 V. 
    Determinar los valores de $L$ y $r$.
\end{problema}


\begin{problema}{}
    En el circuito indicado, la fuente de tensión $E$ entrega 100 V con una 
    frecuencia de 50 Hz y los elementos que lo constituyen son:
    $C = 20$ $\mu$F, $L = 0.25$ H, y $R_1 = R_2 = R_3 = 10$ $\Omega$. 
    \begin{enumerate}[(a)]
        \item Calcular la impedancia equivalente a la derecha de los puntos 
            A y B.
        \item Calcular la corriente que circula por cada resistencia.
        \item Construir el diagrama vectorial del circuito.
\end{enumerate}
\end{problema}

\begin{problema}{}
    Para el circuito de la figura:
    \begin{enumerate}[(a)]
        \item Hallar el valor de la impedancia compleja equivalente.
        \item Determinar su valor en resonancia.
        \item ¿Cuánto vale la frecuencia $\omega$ en este caso?
        \item Construir el diagrama vectorial de la corriente por cada una
            de las ramas.
    \end{enumerate}
\end{problema}

\begin{problema}{}
    Para el circuito de la figura, hallar:
    \begin{enumerate}[(a)]
        \item Las	corrientes	que	circulan por	cada	rama empleando el 
            método de mallas.
        \item La potencia suministrada por cada generador.
        \item La potencia disipada en cada impedancia. \\

    Datos: $V_1 = 30$ V, $V_2 = 20$ V, $Z_1 = 5$ $\Omega$, $Z_2 = 4$ $\Omega$,
    $Z_3  = (2+3j)$ $\Omega$, $Z_4 = 5j$ $\Omega$, $Z_5 = 6$ $\Omega$ y 
    $f = 50$ Hz.
    \end{enumerate}
\end{problema}

\end{document}
