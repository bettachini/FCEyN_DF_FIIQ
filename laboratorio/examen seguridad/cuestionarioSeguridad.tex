\documentclass[11pt,spanish,a4paper,twoside]{article}
% Versión 2do cuat 2015 Víctor Bettachini < bettachini@df.uba.ar >

\usepackage[spanish]{babel}
\addto\shorthandsspanish{\spanishdeactivate{~<>}}
\usepackage[utf8]{inputenc}

\usepackage{float}

\usepackage[separate-uncertainty=true, multi-part-units=single, locale=FR]{siunitx}

\usepackage{amsmath}
\usepackage{amstext}
\usepackage{amssymb}

\usepackage{enumerate}

\newcommand{\pvec}[1]{\vec{#1}\mkern2mu\vphantom{#1}}

\usepackage{tikz}
\usetikzlibrary{decorations.pathmorphing}
\usetikzlibrary{patterns}
% \input{DimLinesTikz}

\usepackage{graphicx}
\graphicspath{ {./graphs/} {../}}
\usepackage{wrapfig}

\voffset-3.5cm
\hoffset-2.5cm
\setlength{\textwidth}{18.5cm}
\setlength{\textheight}{27cm}

\usepackage{lastpage}
\usepackage{fancyhdr}
\pagestyle{fancyplain}
\fancyhead{}
\fancyfoot{{\tiny \textcopyright Departamento de Física, FCEyN, UBA}}
\fancyfoot[C]{ {\tiny Actualizado al \today} }
\fancyfoot[RO, LE]{\tiny Pág. \thepage/\pageref{LastPage}}
\renewcommand{\headrulewidth}{0pt}
\renewcommand{\footrulewidth}{0pt}

\usepackage{multicol}

% textsub super script
\usepackage{changes}


% \usepackage{draftwatermark}
% \SetWatermarkText{Confidencial}
% \SetWatermarkScale{5}

% \usepackage[printwatermark]{xwatermark}
% \newwatermark*[allpages,color=red!50,alpha=0.1,angle=45,scale=3,xpos=0,ypos=0]{Confidencial}

%% biblatex
\usepackage[style=verbose-note, backend=biber, sorting= none]{biblatex}
% \usepackage[style=numeric, backend=biber, sorting= none]{biblatex}
% \usepackage[style=numeric, autocite=footnote, backend=biber, sorting= none]{biblatex}
\DefineBibliographyStrings{spanish}{}
\usepackage{csquotes}
\addbibresource{refParcial.bib}



\begin{document}
\noindent
\textsc{\large Física 2 (Químicos)} \hfill \textsc{\large Cuestionario sobre seguridad en el laboratorio} \\
% \textsc{\large Física 2 (Químicos)} - Prof. Diana Skigin \hfill \textsc{\large Seguridad en el laboratorio} - 2"o cuat. 2016\\

% \noindent Apellido: \dots\dots\dots\dots\dots\dots\dots\dots\dots\dots\dots\dots\dots \hfill L.U. N"o: \dots\dots\dots\dots

% \center \noindent \framebox{Si la hay, citá la referencia en la que se basas tu respuesta.}
% \hline

\noindent
\begin{enumerate}

\item ¿Como me manejo con el maté (termo, yerba, bizcochitos salados, etc.) en el laboratorio? % Reglas I.2 
\vspace{3.5 cm}

\item Se me cae un líquido de un experimento. ¿Le pido al pañolero un paño (¡cuac!) para secarle? % Reglas I.8
\vspace{3.5 cm}

\item En las ``Reglas básicas \ldots'' sección II bajó el título ``De incendios'' punto 7 se indica que debe evacuarse por la ruta designada. ¿Cuál es esa ruta partiendo desde donde estás ahora?
\vspace{3.5 cm}

\item Así como ``lo que mata es la humedad'', ¿que es ``lo peligroso''? ¿La ``alta tensión'' (diferencia de potencial)? ¿O una ``alta'' corriente? ¿Cuanta? % Normas, Shock eléctrico
\vspace{3.5 cm}

\item En las ``Normas de seguridad'' en su sección ``La corriente eléctrica...'' en su última subsección (``Consideraciones \ldots'') se insiste en ``Controlar la calidad de la tierra de su circuito antes de conectarlo'' ¿Que es en particular lo que tengo que ``controlar''? ¿Como me protege ``la tierra''? % Normas, Consideraciones a tener en cuenta antes de empezar a trabar en su experimento
	
\end{enumerate}
\end{document}
