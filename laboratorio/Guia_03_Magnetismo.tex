\documentclass[laboratorio]{guia}

\def \practnum {3} 
\def \practica {Magnetost\'atica: Leyes de Amp\`ere y de Biot-Savart}

\def \materia {Laboratorio de Fisica II para Quimicos}
\def \periodo {2do. Cuatrimestre de 2015}
\def \catedra {P. Cobelli}
 
% \usepackage{graphics}
% \usepackage{amsmath}
\usepackage{amsfonts}
\usepackage{graphicx}
\usepackage[arrowdel]{physics}
\usepackage{float}
\usepackage{wrapfig}
\usepackage{subfigure}
% \usepackage{bm}
% \usepackage{grffile}
\usepackage{color}
% \usepackage{framed}
\usepackage[utf8]{inputenc}
% \usepackage[T1]{fontenc}
% \usepackage{lmodern}

\usepackage[separate-uncertainty=true, multi-part-units=single, locale=FR]{siunitx}

\usepackage{tikz}
\usepackage[siunitx]{circuitikz}

% circuitikz: rotate instruments - https://tex.stackexchange.com/questions/105864/rotate-voltmeter-circuitikz
\newcommand{\mymeter}[2] 
{  % #1 = name , #2 = rotation angle
\begin{scope}[transform shape,rotate=#2]
\draw[thick] (#1)node(){$\mathbf V$} circle (11pt);
\draw[rotate=45,-latex] (#1)  +(-17pt,0) --+(17pt,0);
\end{scope}
}

\usepackage{isotope} % $\isotope[A][Z]{X}\to\isotope[A-4][Z-2]{Y}+\isotope[4][2]{\alpha}$

\usepackage[spanish]{babel}
\usepackage{babelbib}
\selectbiblanguage{spanish}
 
% definicion del entorno 'sabermas'
\makeatletter
\definecolor{shadecolor}{rgb}{0.89,0.91,0.94}
\newenvironment{sabermas}[1]{%
\vfill
\begin{shaded}
  \begin{center}
  {\textsection{Para saber m\'as}}
  \end{center}
  #1
\sf } 
{%
\end{shaded}%
}
\makeatother

\renewcommand{\vec}[1]{\ensuremath{\mathbf{#1}}}



\input{hyphenation_rules}
% \graphicspath{{./guia8/}}

\begin{document} 
\objetivo{%
    El objetivo de esta gu\'\i a experimental consiste en determinar
    cuantitativamente el valor y direcci\'on del campo magn\'etico terrestre
    local (en el laboratorio) mediante dos m\'etodos experimentales diferentes.
    \tematicas{Campo magn\'etico, magnetost\'atica, ley de Amp\`ere, ley de
        Biot-Savart, sonda Hall.}} 
\maketitle

\section{Medici\'on de campos magn\'eticos usando una sonda Hall}

Las sondas (o puntas) Hall son dispositivos basados en el efecto Hall que
permiten medir con gran precisi\'on la componente del campo magn\'etico
perpendicular a su plano de trabajo. El voltaje Hall se mide mediante un
volt\'\i metro. 

En el caso concreto de las sondas Hall disponibles en el laboratorio, su uso
requiere conectar la sonda al puerto USB de la PC para alimentar el
amplificador de la misma. El amplificador se utiliza para aumentar la tensi\'on
de salida de la sonda Hall (que es del orden de los 3 mV/Gauss) a un rango
de 0 a 4 V. 

Conecte la sonda Hall y familiar\'\i cese con su funcionamiento y su
operaci\'on. Consulte a su docente si tiene dudas respecto de cualquiera de
estos dos aspectos. Una vez que comprenda c\'omo funciona la sonda, explore la
respuesta de la misma variando su orientaci\'on espacial. De esta forma,
encuentre el plano de trabajo de la misma. 

\section{Calibraci\'on de la sonda Hall}

Sabiendo que el campo magn\'etico en el interior de una bobina por la cual
circula una corriente cont\'\i nua es uniforme, podemos emplear este campo para 
realizar la calibraci\'on de la sonda Hall (?`por qu\'e raz\'on es necesario
calibrar la sonda?). 

Usando una fuente de tensi\'on, una resistencia variable, una bobina de
geometr\'\i a y n\'umero de vueltas conocidas y un amper\'\i metro, arme un
circuito de modo que pueda -simult\'aneamente- aplicar y medir la corriente que
circula por la bobina. 

Dise\~ne un montaje que le permita mantener la sonda Hall en el centro de la
bobina. El m\'odulo del campo magn\'etico en el centro de la bobina de radio $R$, longitud
$L$, n\'umero de vueltas $N$, por la que circula una corriente $I$ puede
aproximarse por:
\begin{equation}
    |\vec{B}| = \alpha \frac{\mu_0}{2R} N I,
    \end{equation}
siendo $\alpha$ un factor de proporcionalidad que depende de las caracter\'\i
sticas geom\'etricas detalladas de la bobina, y que en nuestro caso puede
aproximarse por $\alpha \approx 0.28$ (verificar en cada caso).

Con estos datos, calibre la sonda Hall utilizando como patr\'on el campo
magn\'etico de la bobina.


\section{Determinaci\'on del campo magn\'etico terrestre}

Una vez calibrada la sonda Hall, se proponen las siguientes dos actividades
para determinar el campo magn\'etico terrestre:
\begin{itemize}
    \item Mida el campo magn\'etico terrestre empleando la sonda Hall.
        Explore tambi\'en qu\'e ocurre cuando rota la punta en 180$^\circ$
        respecto de su eje, de modo que el campo magn\'etico atraviese la punta
        Hall por el extremo opuesto.

    \item Ubique una br\'ujula en el centro de la bobina empleada
        anteriormente. Con la bobina sin corriente determine la direcci\'on del
        campo magn\'etico terrestre. Aseg\'urese de alinear correctamente la
        bobina, de modo que su eje quede perpendicular a la direcci\'on del
        campo magn\'etico terrestre local. Luego haga pasar una corriente por
        las espiras y determine la dependencia del \'angulo que se desv\'\i a
        la aguja (que llamaremos $\theta$) con el campo $\vec{B}$ de la bobina. A partir de la medici\'on
        del \'angulo de desviaci\'on, y conociendo el campo generado por la
        bobina, se puede determinar el campo terrestre.

    \item Grafique $B_\text{bobina}$ vs $\tan \left( \theta \right)$, empleando
        para ello no menos de 6 puntos. ?`Qu\'e an\'alisis puede realizar a
        partir de esta gr\'afica?
\end{itemize}

\nocite{Alonso1998,Purcell1988,Reitz1996,Trelles1984}
\bibliographystyle{unsrt} 
\bibliography{Bibliografia}

\end{document}
