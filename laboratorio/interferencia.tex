\documentclass[laboratorio]{guia}

\def \practnum {8}
\def \practica {Interferencia}
% \def \practica {Interferencia: Biprisma de Fresnel}

\def \materia {Laboratorio de Fisica II para Quimicos}
\def \periodo {2do. Cuatrimestre de 2015}
\def \catedra {P. Cobelli}

% \usepackage{graphics}
% \usepackage{amsmath}
\usepackage{amsfonts}
\usepackage{graphicx}
\usepackage[arrowdel]{physics}
\usepackage{float}
\usepackage{wrapfig}
\usepackage{subfigure}
% \usepackage{bm}
% \usepackage{grffile}
\usepackage{color}
% \usepackage{framed}
\usepackage[utf8]{inputenc}
% \usepackage[T1]{fontenc}
% \usepackage{lmodern}

\usepackage[separate-uncertainty=true, multi-part-units=single, locale=FR]{siunitx}

\usepackage{tikz}
\usepackage[siunitx]{circuitikz}

% circuitikz: rotate instruments - https://tex.stackexchange.com/questions/105864/rotate-voltmeter-circuitikz
\newcommand{\mymeter}[2] 
{  % #1 = name , #2 = rotation angle
\begin{scope}[transform shape,rotate=#2]
\draw[thick] (#1)node(){$\mathbf V$} circle (11pt);
\draw[rotate=45,-latex] (#1)  +(-17pt,0) --+(17pt,0);
\end{scope}
}

\usepackage{isotope} % $\isotope[A][Z]{X}\to\isotope[A-4][Z-2]{Y}+\isotope[4][2]{\alpha}$

\usepackage[spanish]{babel}
\usepackage{babelbib}
\selectbiblanguage{spanish}

% definicion del entorno 'sabermas'
\makeatletter
\definecolor{shadecolor}{rgb}{0.89,0.91,0.94}
\newenvironment{sabermas}[1]{%
\vfill
\begin{shaded}
  \begin{center}
  {\textsection{Para saber m\'as}}
  \end{center}
  #1
\sf } 
{%
\end{shaded}%
}
\makeatother

\renewcommand{\vec}[1]{\ensuremath{\mathbf{#1}}}


\input{hyphenation_rules}
\graphicspath{{./interferencia/}}

\begin{document}
\objetivo{%
  Determinar la longitud de onda de mayor intensidad emitida por una lámpara de Sodio, empleando para ello un método interferométrico.
  \tematicas{interferencia, dos rendijas, biprisma de Fresnel, espectro de emisión del Sodio, doblete del Sodio.}
  }
\maketitle


\section{Introducción}

El biprisma de Fresnel es un interferómetro por división de frente de onda similar al experimento de la doble rendija de Young.
El mismo consta de dos prismas delgados que sirven para generar dos imágenes coherentes de una única fuente (una rendija iluminada) de modo tal que la luz proveniente de ambas da lugar a interferencia en la zona situada a continuación del biprisma.
Estas franjas son franjas reales \emph{no localizadas}, es decir que pueden verse sobre una pantalla en toda una región que se extiende más allá de la posición del biprisma.
Se puede demostrar que el plano donde se encuentran ubicadas las fuentes virtuales generadas por el biprisma es el mismo plano en el cual se encuentra dispuesta la rendija. 

En cada punto del espacio donde la diferencia de camino óptico (\(\Delta \Lambda\)) entre las ondas provenientes de cada fuente resulte igual a un número entero (\(n,\, n \in \mathbb{N}_0\)) de longitudes de onda (\(\Delta \Lambda= n \lambda\)) habrá entonces interferencia constructiva y se verá una franja brillante. 

Más aún, es posible mostrar que la separación entre franjas de interferencia viene dada por
\begin{equation}
    \delta y = s \frac{\lambda}{a},
    \label{eq:1}
\end{equation}
donde \(\delta y\) representa la distancia entre dos máximos brillantes consecutivos (interfranja), \(s\) corresponde a la distancia entre el plano de las fuentes virtuales y el plano donde se observa el patrón de interferencia, y \(a\) denota la distancia entre las dos fuentes virtuales responsables por la interferencia, como se esquematiza en la figura \ref{fig:biprisma}.

\begin{figure}[ht]
    \centering
    \includegraphics[width=8.5cm]{LG09--000.png}
    \caption{Esquema del montaje sugerido para observar el fenómeno de interferencia entre dos fuentes puntuales (virtuales).}
    \label{fig:biprisma}
\end{figure}


El propósito de la presente experiencia es el de determinar experimentalmente, a partir de la ecuación \eqref{eq:1}, la longitud de onda de mayor intensidad emitida por una lámpara de Sodio (\isotope{Na}), que a los efectos prácticos de esta guía se considerará monocromática.


\section{Desarrollo de la experiencia}

\subsection{Calibración del microscopio y medición de la distancia de enfoque}

Dado que la figura de interferencia es muy pequeña, la misma no puede observarse a simple vista, por lo que se requiere del uso de un microscopio de banco.
Este elemento óptico cuenta con un retículo que puede desplazarse empleando una rueda graduada (micrómetro), a fin de poder medir los objetos que yacen dentro de su campo visual.

Antes de poder realizar dichas mediciones es necesario determinar la escala en la que está graduado el micrómetro. Para ello, enfoque con el microscopio un papel milimetrado y determine experimentalmente a qué distancia equivale una unidad sobre la escala del micrómetro.

Preste particular atención al hecho de que el micrómetro del microscopio \textbf{solo debe rotarse en el mismo sentido}, (sea cual fuere éste), dado que el mismo no vuelve sobre sus mismos pasos. 


\subsection{Medición de la longitud de onda de la lámpara de Sodio}

Sobre un banco óptico, coloque el microscopio, el biprisma y la rendija.
Note que puede resultarle conveniente ubicar el biprisma en un brazo con desplazamiento lateral (perpendicular al eje del banco), dado que esta práctica presenta -como requisito fundamental- disponer de los elementos ópticos correctamente alineados. 

Por otro lado, la lámpara de \isotope{Na} requiere de un tiempo para entrar en régimen, por lo que conviene prenderla varios minutos antes de comenzar; ya que de otro modo se estarían viendo longitudes de onda que no son las que se busca determinar.

Teniendo en cuenta estas precauciones, comience la experiencia.
Para ello, acerque el microscopio al biprisma y desplace éste lateralmente hasta ver ambas fuentes virtuales, las cuales deben aparecer de igual intensidad, ancho y altura.
Desenfoque entonces ligeramente el microscopio y asegúrese de que aún en este caso ambas fuentes virtuales continúan presentando las mismas características.
Aleje el microscopio a una distancia donde pueda observar ahora las franjas de interferencia.


\nocite{Alonso1998,Jenkins2001,Hecht1986}
\bibliographystyle{unsrt} 
\bibliography{Bibliografia}
        

\end{document}
